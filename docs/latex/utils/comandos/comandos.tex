% ----------------------------------------------------------------------------------------------------
% ----------------------------------------------------------------------------------------------------
% All new command documantation can be found in utils/comandos_doc.tex
% ----------------------------------------------------------------------------------------------------
% ----------------------------------------------------------------------------------------------------

% Multiline comment in Latex script. 
% ----------------------------------------------------------------------------------------------------
\newcommand{\multilinecomment}[1]{}

% Table macro insertion.
% ----------------------------------------------------------


\newcommand{\simpletable}[7][c]{%
    \begin{table}[h!]
            \centering
            \caption{\uppercase{#3}}
            \vspace{2pt}
            \begin{tabular}[#1]{#2}
                \hline
                {#4}
                \label{tab:#5}
            \end{tabular}
            \\
            \begin{minipage}{#7 \textwidth}
                \centering
                \vspace{6pt}
                FONTE:~\textcite{#6}
            \end{minipage}
    \end{table}
}
        
% Figure insertion macro
% ----------------------------------------------------------

\newcommand{\figura}[5]{%
    \begin{center}\small
        \begin{figure}[H] % Places the float at precisely the location in the LATEX code
            \centering
                \uppercase{\caption{\uppercase{#1}}} % Define figure title
                \includegraphics[width=#2\textwidth]{#3} % Define figure proportion and the figure path
                \label{fig:#4}
            %ajustado p/ a largura da imagem
            \begin{minipage}{#2\textwidth} 
                \vspace{0mm}
                \centering
                    \par FONTE:~#5 % Define the figure source
            \end{minipage}
        \end{figure}
    \end{center}
}

% Quadro insertion macro
% ----------------------------------------------------------
%
\newcommand{\figuraquadro}[5]{%
    \begin{center}\small%
        \begin{quadro}[H] % Places the float at precisely the location in the LATEX code
            \centering%
                \uppercase{\caption{\uppercase{#1}}} % Define quadro title
                \includegraphics[width=#2\textwidth]{#3} % Define figure proportion and the figure path
                \label{tab:#4}%
            %ajustado p/ a largura da imagem
            \begin{minipage}{#2\textwidth}%
                \vspace{0mm}%
                \centering%
                    \par FONTE:~#5 % Define the figure source
            \end{minipage}%
        \end{quadro}%
    \end{center}%
}%
%
% Quadro insertion macro (tabular)
% ----------------------------------------------------------

\newcommand{\figuraquadrotabular}[5]{%
\noindent
\begin{center}\small
    \begin{quadro}[h!] % Places the float at precisely the location in the LATEX code
        \centering
            \uppercase{\caption{\uppercase{#1}}} % Define quadro title
            \vspace{2mm}
            {#2} % Define figure proportion and the figure path
            \label{tab:#3}
        %ajustado p/ a largura da imagem
        \begin{minipage}{#4\textwidth} 
            \vspace{2mm}
            \centering
                \par FONTE:~#5 % Define the figure source
        \end{minipage}
    \end{quadro}
\end{center}
}


\newcommand{\citeapud}[2]{(\Citealp{#1} apud \Citealp{#2})}