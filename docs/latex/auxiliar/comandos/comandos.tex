% ----------------------------------------------------------------------------------------------------
% ----------------------------------------------------------------------------------------------------
% All new command documantation can be found in utils/comandos_doc.tex
% ----------------------------------------------------------------------------------------------------
% ----------------------------------------------------------------------------------------------------

% Multiline comment in Latex script. 
% ----------------------------------------------------------------------------------------------------
\newcommand{\multilinecomment}[1]{}

% Table macro insertion.
% ----------------------------------------------------------
\newcommand{\tabelasimples}[5]{%
	\begin{table}[H]
		\centering
		\small
		\def\arraystretch{1.2}
        \caption{\centering\uppercase{#2}}
        \vspace{2pt}
        \begin{tabular}[c]{#1}
        	\hline
            #3 %
        \end{tabular}
        \begin{minipage}{1\textwidth}
            \vspace{6pt}
            \centering
            \par FONTE:~#5
            \label{tab:#4}
        \end{minipage}
    \end{table}
}

% Table macro insertion.
% ----------------------------------------------------------
\newcommand{\tabelamultilinhas}[5]{%
	\begin{table}[H]
		\centering
		\small
		\def\arraystretch{1.2}
		\caption{\centering\uppercase{#2}}
		\vspace{2pt}
		\begin{tabular}[c]{#1}
			#3 %
		\end{tabular}
		\begin{minipage}{1\textwidth}
			\vspace{6pt}
			\centering
			\par FONTE:~#5
			\label{tab:#4}
		\end{minipage}
	\end{table}
}

% Table macro insertion.
% ----------------------------------------------------------
\newcommand{\tabelamulticolunas}[5]{%
	\begin{table}[H]
		\centering
		\small
		\def\arraystretch{1.2}
		\caption{\centering\uppercase{#2}}
		\vspace{2pt}
		\begin{tabular}[c]{#1}
			#3 %
		\end{tabular}
		\begin{minipage}{1\textwidth}
			\vspace{6pt}
			\centering
			\par FONTE:~#5
			\label{tab:#4}
		\end{minipage}
	\end{table}
}

% Table macro insertion.
% ----------------------------------------------------------
\newcommand{\tabela}[4]{%
	\begin{table}[H]
		\centering
		\small
		\def\arraystretch{1.2}
		\caption{\centering\uppercase{#1}}
		\vspace{2pt}
		{#2}
		\begin{minipage}{1\textwidth}
			\vspace{6pt}
			\centering
			\par FONTE:~#4
			\label{tab:#3}
		\end{minipage}
	\end{table}
}

% Table macro insertion.
% ----------------------------------------------------------
\newcommand{\tabelalonga}[3]{%
	\begin{longtable}{#1} \hline
		\centering
		\small
		\def\arraystretch{2}
		#2
	\end{longtable}
}

% Figure insertion macro
% ----------------------------------------------------------

\newcommand{\figura}[5]{%
    \begin{center}
    	\small
        \begin{figure}[H] % Places the float at precisely the location in the LATEX code
            \centering
                \uppercase{\caption{\centering\uppercase{#1}}} % Define figure title
                \includegraphics[width=#2\textwidth]{#3} % Define figure proportion and the figure path
            %ajustado p/ a largura da imagem
            \begin{minipage}{#2\textwidth} 
                \vspace{0mm}
                \centering
                \par FONTE: ~#5 % Define the figure source
                \label{fig:#4}
            \end{minipage}
        \end{figure}
    \end{center}
}

\newcommand{\figuracomnota}[6]{%
	\begin{center}
		\small
		\begin{figure}[H] % Places the float at precisely the location in the LATEX code
			\centering
			\uppercase{\caption{\centering\uppercase{#1}}} % Define figure title
			\includegraphics[width=#2\textwidth]{#3} % Define figure proportion and the figure path
			%ajustado p/ a largura da imagem
			\begin{minipage}{#2\textwidth}
				\vspace{0mm}
				{\centering FONTE: ~#5 \par } % Define the figure source
				\hspace{20pt} NOTA: ~#6 \par % Define the figure extra information
			\end{minipage}
		\label{fig:#4}
		\end{figure}
	\end{center}
}

% Citação
% ----------------------------------------------------------
\newcommand{\citacao}[4]{%
	\vspace{0.5cm}
	\hspace{4cm}
	\begin{minipage}{0.65\textwidth}
		\scriptsize
		\setlength{\parindent}{0em}
		#1~#3~#4
	\end{minipage}
	\footnote{\textit{#2}}
	\vspace{0.5cm}
}

% Citação apud
% ----------------------------------------------------------
\newcommand{\citeapud}[2]{%
	(\Citealp{#1} apud \Citealp{#2})
}

\newcommand{\citeapudyear}[2]{%
	(\citeyear{#1} apud \Citealp{#2})
}

\renewcommand{\footnotesize}{%
	\fontsize{9pt}{10pt}\selectfont
}
