\chapter{Aplicação de Algoritmos Genéticos na Economia}

Neste capítulo, busca-se apresentar algumas aplicações de AGs em problemas na economia e discussões voltadas a contextos econômicos, especialmente, nas áreas de otimização, inovação e aprendizagem adaptativa.

\section{Introdução}

A economia, sendo uma ciência social, não possui um laboratório totalmente controlado para se testar hipóteses e modelos, havendo a necessidade de se utilizar premissas que tornem possível a aplicação matemática de conceitos e modelos teóricos em busca de encontrar respostas para as perguntas a serem respondidas sobre o comportamento dos agentes econômicos e da sociedade como um todo. Contudo, muitas vezes, os modelos se apresentam limitados ou insuficientes quando testados em problemas reais complexos.

A gama de áreas que compõe a economia no que diz respeito ao escopo de análises dos agentes, políticas públicas, conjecturas, entre outras, é vasta. Porém, pode-se dividi-la em dois ramos principais: macroeconomia e microeconomia. Conforme apresenta \citet[pg.3]{pindyck_microeconomia_2013}, a microeconomia estuda os agentes a nível individual, sendo trabalhadores, consumidores, investidores, empresas, etc. Neste nível, analisa-se como os agentes que participam dos processos econômicos tomam suas decisões, fazem suas escolhas ou interagem entre si, formando agentes econômicos maiores que apenas uma unidade isolada. Assim sendo, a microeconomia não busca observar indicadores agregados como a taxa de crescimento da economia, a inflação, o nível de desemprego, entre outros. Cabe à macroeconomia a análise destes agregados, e será nela que iremos nos concentrar para entender as formas de aplicação dos AGs em contextos econômicos.

O escopo de estudos da macroeconomia é apresentado por \citeauthor*{snowdon_modern_2005} da seguinte maneira:

\citacao
	{%
		A macroeconomia está preocupada com a estrutura, desempenho e comportamento da economia como um todo. A principal preocupação dos macroeconomistas é analisar e tentar entender os determinantes subjacentes das principais tendências agregadas da economia em relação à produção total de bens e servições (PIB), desemprego, inflação e transações nacionais. Em particular, a análise macroeconômica procura explicar a causa e o impacto das flutuações de curto prazo no PIB (o ciclo de negócios) e os principais determinantes da trajetória de longo prazo do PIB (crescimento econômico)
	}{%
		Macroeconomics is concerned with the structure, performance and behaviour of the economy as a whole. The prime concern of macroeconomists is to analyse and attempt to understand the underlying determinants of the main aggregate trends in the economy with respect to the total output of goods and services (GDP), unemployment, inflation and international transactions. In particular, macroeconomic analysis seeks to explain the cause and impact of short-run fluctuations in GDP (the business cycle), and the major determinants of the long-run path of GDP (economic growth).
	}
	{\citep[pg.1]{snowdon_modern_2005}}
	{(tradução nossa).}

Contudo, claro, os indicadores-objetivo da macroeconomia tem grande importância e impacto na vida das pessoas a nível individual. Uma economia que apresenta uma taxa de crescimento contínuo, e uma baixa taxa de desemprego e inflação, pode ser considerada bem sucedida em relação às suas políticas e estratégias econômicas que, por sua vez, impactarão positivamente o nível de felicidade e bem-estar da população como um todo. Em contrapartida, crises econômicas aumentam a insegurança e incerteza, assim como reduzem o nível geral de bem-estar e felicidade dos indivíduos. Dessa forma, as análises econômicas seguiram separadamente por um longo período, com uma alocação muito clara do que seria uma análise voltada à microeconomia ou macroeconomia, sendo a união dos conhecimentos e teorias destes dois campos de estudo relativamente recente (ver \autoref{sec:AsPrincipaisTeoriasEconomicasDoSeculoXXeXXI}).

Em termos de análises macroeconômicas, um dos principais objetivos, senão o principal, é a elaboração de políticas que visam manter uma estabilidade econômica frente aos choques endógenos e exógenos que podem afetar negativamente um conjunto de indicadores. Busca-se entender como determinados agregados, que compõem uma nação, se comportaram durante um período e como prevê-los no longo prazo como objetivo de antecipar diferentes tipos de cenários. Não há, contudo, uma forma correta, muito menos um consenso, de como se realizar tais análises, criando-se, ao longo do tempo, diferentes metodologias de como construir os argumentos em torno das análises conjecturais do passado, presente ou futuro. \citeauthor*{snowdon_modern_2005} ilustram muito bem a constituição de uma teoria econômica, onde, \enquote{organizada através de uma estrutura lógica (ou teórica), forma a base sobre a qual uma política econômica é projetada e implementada}, sendo que \enquote{[...] o problema intelectual para os economistas é como capturar, na forma de modelos específicos, o complicado comportamento interativo de milhões de indivíduos engajados na atividade econômica}\footnote{\textit{Macroeconomic theory, consisting of a set of views about the way the economy operates, organized within a logical framework (or theory), forms the basis upon which economic policy is designed and implemented. Theories, by definition, are simplifications of reality. This must be so given the complexity of the real world. The intellectual problem for economists is how to capture, in the form of specific models, the complicated interactive behaviour of millions of individuals engaged in economic activity.}} \citep[pg.4]{snowdon_modern_2005} (tradução nossa).

Assim, parte-se, inicialmente, de modelos simples, generalizados e com uma série de hipóteses em busca de cobrir toda a complexidade da realidade. Por sua vez, embora simples, estes modelos apresentam as primeiras respostas para o entendimento de como determinados fenômenos ocorrem e qual seu impacto na sociedade como um todo, podendo se considerar uma teoria bem sucedida, a que apresenta um conjunto de modelos que preveem determinados cenários e quais políticas econômicas poderão ser implementadas visando alcançar objetivos definidos de uma nação que, de maneira prática, apresentam-se como a busca por uma taxa de crescimento econômico satisfatória sem a desestabilização da economia durante o processo.

\section{As Principais Teorias Macroeconômicas do Século XX e XXI}
\label{sec:AsPrincipaisTeoriasEconomicasDoSeculoXXeXXI}

Em torno da pluralidade dos pensamentos e teorias econômicas ao longo do século XX e XXI, \citeauthor*{blanchard_macroeconomics_2020} \citeapudyear{blanchard_macroeconomics_2020}{snowdon_modern_2005}aponta que \enquote{o que os economistas acreditam hoje, é o resultado de um \textbf{processo evolutivo} (destacado por nós) no qual eles eliminaram aquelas ideias que falharam e mantiveram as que parecem explicar bem a realidade.}\footnote{\textit{What macroeconomists believe today is the result of an evolutionary process in which they have eliminated those ideas that failed and kept those that appear to explain reality well.}}

Há um certo consenso de que o entendimento da macroeconomia, como uma área "formal" de estudos, acontece na primeira metade do século XX. Contudo, por todo o período seguinte, houve discordâncias e controvérsias acerca de qual seria a melhor e mais correta metodologia para análises macroeconômicas. \Citet[pg.1-14]{mayer_1994}, em seu artigo intitulado \enquote{\textit{Why is there so much disagreement among economists?}}, lista sete causas principais para a falta de alinhamento entre os macroeconomistas ao longo da história: i) um conhecimento limitado do funcionamento da economia, ii) o conjunto de assuntos investigados em constante crescimento, iii) a necessidade de considerar demais fatores que causam grande impacto na economia que subdivide-se em mais quatro causas como iv) os fatores políticos; v) os julgamentos de valor; vi) as empatias sociais; e, por último, vii) as metodologias.

Ademais, como argumentam \Citet[pg.6-7]{snowdon_modern_2005}, a própria rigidez dos métodos de pesquisa científica convergem para um contexto de discussões e discordâncias, evitando que haja, por exemplo, uma estagnação no campo dos estudos macroeconômicos, assim como os embates se concentram mais nas questões teóricas, evidências apresentadas, ou empíricas, e das ferramentas utilizadas para resolução dos problemas, do que sobre quais são os reais objetivos das políticas econômicas (alto nível de emprego, estabilização dos preços e rápido crescimento econômico). Dessa forma, a escolha das ferramentas ou dos métodos que serão utilizados para resolver estes problemas dependerão dos diagnósticos das causas raízes dos problemas em uma economia. Neste ponto, divide-se os estudos em duas abordagens principais: clássica e keynesiana.

\subsection{Abordagem Clássica}

Dentre as diversas ideias apresentadas por \Citet{smith1996riqueza} em sua obra primordial da, futuramente conhecida como abordagem clássica, \enquote{A Riqueza das Nações: investigação sobre sua natureza e suas causas}, o \text{laissez-faire}\footnote{Expressão francesa que significa, de forma literal, \enquote{deixe estar ou deixe fazer}.} é a principal. Smith defendia que, em um cenário ideal, o governo deveria intervir o mínimo possível nos interesses e interações dos agentes privados, sobretudo, o comércio. Cabia ao governo, na figura do soberano à época, apenas três, importantes, deveres: i) \enquote{[...] o dever de proteger a sociedade contra a violência e a invasão de outros países independentes}; ii) \enquote{[...] o dever de proteger, na medida do possível, cada membro da sociedade contra a injustiça e a opressão de qualquer outro membro da mesma, ou seja, o dever de implementar uma administração judicial exata}; iii) \enquote{[...] o dever de criar e manter certas obras e instituições públicas que jamais algum indivíduo ou um pequeno contingente de indivíduos poderão ter o interesse de criar e manter [...]} \cite[pg.170]{smith2020riquezav2}.

Como apontado por \Citet[pg.105]{vasconcellos2015}, o período dos estudos econômicos referente à Escola Clássica (pré Keynes) é formado pela contribuição de diversos autores que contribuíram de forma isolada ao longo do século XIX e início do XX, tendo, em grande parte, as ideias de Smith como precursoras. Assim sendo, quando se fala da abordagem clássica, ou Escola Clássica, entende-se como uma compilação das principais contribuições pré keynesianismo.

Para a Escola Clássica, o mercado tende a sempre alcançar o equilíbrio econômico através do pleno emprego, ou seja, onde a oferta e a demanda por mão-de-obra são iguais. Assim, como existe uma predeterminação do nível de atividade e de emprego pelas forças de mercado, as chamadas variáveis reais, a variação dos preços é impactada apenas pela quantidade de moeda, isto é, não afeta os agregados econômicos nem os preços relativos. Com isso, valendo-se da máxima \enquote{a oferta cria sua própria demanda} (Lei de Say), as políticas monetárias não têm efeito sobre a oferta e demanda agregada (hipótese conhecida como a neutralidade da moeda\footnote{Na abordagem clássica há uma separação entre as variáveis reais (nível de emprego, produto, salário real, etc.) e as variáveis nominais ou monetárias (preços e salário nominal). A neutralidade da moeda parte da hipótese de que o estoque de moeda não afeta as variáveis reais, algo que ocorrerá apenas se existir alguma imperfeição ou distorção nos mercados.}), pois a demanda agregada não é um fator de impacto quando se olha para a oferta agregada que, por sua vez, é determinado pela produção total que as empresas e as famílias estão dispostas a ofertar por um determinado preço.

Dessa forma, partindo do pressuposto de que o mercado de trabalho é do tipo concorrência perfeita, ou seja, com um grande número de ofertantes (sem fixação dos preços por parte de sindicatos, por exemplo) e um grande número de demandantes (sem fixação dos salários que serão pagos pela empresa), o equilíbrio no mercado de trabalho é alcançado quando a demanda por mão-de-obra se iguala à oferta por mão-de-obra. Em outras palavras, num dado nível de salário real, a empresa encontrará uma oferta de trabalho suficiente para suprir a demanda e todos os trabalhadores que quiserem trabalhar, encontrarão um emprego. No caso de um excesso na oferta de trabalho, o resultado será a queda do salário real ($\left(\frac{W}{P}\right)_{1}$ para $\left(\frac{W}{P}\right)_{E}$) e um excesso na demanda de trabalho resultará no aumento do salário real ($\left(\frac{W}{P}\right)_{2}$ para $\left(\frac{W}{P}\right)_{E}$). Assim, o mercado sempre forçará um equilíbrio entre a oferta e a demanda por mão-de-obra (vide \autoref{fig:AbordagemClassicaEquilibrioMercadoTrabalho}).

\figuracomnota
	{Equilíbrio no Mercado de Trabalho}
	{.8}
	{imagens/AbordagemClassicaEquilibrioMercadoTrabalho.PNG}
	{AbordagemClassicaEquilibrioMercadoTrabalho}
	{adaptado de \citet[p.113]{vasconcellos2015}}
	{$N$: oferta de trabalho; $\frac{W}{P}$: salário real;}

Então, definido o nível de emprego de equilíbrio do mercado de trabalho e um determinado produto de pleno emprego, as variáveis que irão impactar na oferta agregada serão apenas as variáveis reais, ou seja, os fatores que, quando alterados, podem impactar no mercado de trabalho, como o aumento ou queda na produção marginal do trabalho (variação do acúmulo de capital ou na tecnologia), que pode afetar a demanda por mão-de-obra, ou um aumento populacional, por exemplo, que afeta a oferta por mão-de-obra. Em relação à demanda agregada, a oferta de moeda tenderá a ser sempre igual à demanda de moeda que, por sua vez, é proporcional à quantidade do produto total. Assim, para uma determinada quantidade de moeda ofertada, quanto maior o nível de preços, menor será o estoque de moeda disponível para satisfazer as transações em uma economia e, por consequência, também será menor a quantidade de bens e serviços a serem demandados. No caso de um aumento no estoque de moeda, a demanda agregada aumentará, e, no caso de uma queda, a demanda agregada irá reduzir.

Com isso, temos que o produto total de uma economia é afetado apenas pelas variáveis que influenciam a oferta agregada e que os fatores que influenciam a demanda agregada não impactam no produto total, já que afetam apenas os preços. O equilíbrio de pleno emprego, então, é alcançado através dos mecanismos de preços que operam a auto-regulação do mercado, evitando, assim, uma intervenção governamental em busca do equilíbrio.

\subsection{Abordagem Keynesiana}

A abordagem keynesiana é, sobretudo, uma negação da abordagem clássica. Como cita \citeauthor*{mankiw1989}, a ideia da auto-regulação dos agentes privados e intervenção mínima por parte do Estado frente às novas ideias apresentadas por Keynes de uma maior intervenção estatal em busca de regular a demanda agregada, visando manter um certo nível de produção e emprego, dominou as discussões econômicas ao longo do século XIX e XX:

\citacao
{%
	A Escola Clássica enfatiza a otimização dos atores econômicos privados, o ajuste dos preços para o equilíbrio da oferta e demanda e a eficiência dos mercados livres. A Escola Keynesiana acredita que o entendimento das flutuações econômicas requer, não apenas estudar os meandros do equilíbrio geral, mas também apreciar a possibilidade de falha de mercado em grande escala.
}{%
	The classical school emphasizes the optimization of private economic actors, the adjustment of relative prices to equate supply and demand, and the efficiency of unfettered markets. The Keynesian school believes that understanding economic fluctuations requires not just studying the intricacies of general equilibrium, but also appreciating the possibility of market failure on a grand scale.
}
{\citep[pg.79]{mankiw1989}}
{(tradução nossa).}

Segundo \Citet[pg.14]{snowdon_modern_2005}, Keynes via a teoria clássica como o grande problema da ineficiência de análises econômicas, à época, na resolução das dores sofridas pelas sociedades ao redor do mundo no período entre guerras, sendo desastrosa quando aplicada em problemas do mundo real. Para ele, o capitalismo não era a causa raiz, embora se apresentava como um sistema instável. Com isso, Keynes propôs novas ideias visando fortalecer este sistema por meio da diminuição ou eliminação completa das falhas mais pungentes do capitalismo através da busca pelo pleno emprego e estabelecendo as questões de eficiência, estabilidade e crescimento econômico como os temas centrais das políticas econômicas.

\section{Economia Evolucionária}

\section{Algoritmos Genéticos Aplicados a Problemas de Otimização}

\section{Algoritmos Genéticos Aplicados a Problemas de Inovação}

\section{Aprendizagem Adaptativa}

