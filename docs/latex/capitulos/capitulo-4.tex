\chapter{Aplicação de Algoritmos Genéticos na Economia}

Neste capítulo, busca-se apresentar algumas aplicações de AGs em problemas na economia e discussões voltadas a contextos econômicos, especialmente, nas áreas de otimização, inovação e aprendizagem adaptativa.

\section{Introdução}

A economia, sendo uma ciência social, não possui um laboratório totalmente controlado para se testar hipóteses e modelos, havendo a necessidade de se utilizar premissas que tornam possível a aplicação matemática de conceitos e modelos teóricos em busca de encontrar respostas para as perguntas a serem respondidas sobre o comportamento dos agentes econômicos e da sociedade como um todo. Contudo, muitas vezes, os modelos se apresentam limitados ou insuficientes quando testados em problemas reais complexos.

A gama de áreas de estudo que compõe a economia  no que diz respeito ao escopo de análises de agentes, políticas públicas, conjecturas, entre outras, é vasta. Contudo, pode-se dividi-la em dois ramos principais: a macroeconomia e a microeconomia. Conforme apresenta \citet[pg.3]{pindyck_microeconomia_2013}, a microeconomia estuda os agentes a nível individual, sendo trabalhadores, consumidores, investidores, empresas, etc. Nesse nível, analisa-se como os agentes que participam dos processos econômicos tomam suas decisões, fazem suas escolhas ou interagem entre si, formando agentes econômicos maiores que apenas a unidade analisada. De forma geral, a microeconomia busca observar indicadores não agregados, como a taxa de crescimento de um país, inflação, nível de desemprego, entre outros. De forma geral, cabe à macroeconomia a análise destes agregados, e será nela que iremos nos concentrar para entender as formas de aplicação dos AGs em contextos econômicos.

O escopo de estudos da macroeconomia é apresentado por \citeauthor*{snowdon_modern_2005} da seguinte maneira:

\citacao
	{%
		A macroeconomia está preocupada com a estrutura, desempenho e comportamento da economia como um todo. A principal preocupação dos macroeconomistas é analisar e tentar entender os determinantes subjacentes das principais tendências agregadas da economia em relação à produção total de bens e servições (PIB), desemprego, inflação e transações nacionais. Em particular, a análise macroeconômica procura explicar a causa e o impacto das flutuações de curto prazo no PIB (o ciclo de negócios) e os principais determinantes da trajetória de longo prazo do PIB (crescimento econômico)
	}{%
		Macroeconomics is concerned with the structure, performance and behaviour of the economy as a whole. The prime concern of macroeconomists is to analyse and attempt to understand the underlying determinants of the main aggregate trends in the economy with respect to the total output of goods and services (GDP), unemployment, inflation and international transactions. In particular, macroeconomic analysis seeks to explain the cause and impact of short-run fluctuations in GDP (the business cycle), and the major determinants of the long-run path of GDP (economic growth).
	}
	{\citep[pg.1]{snowdon_modern_2005}}
	{(tradução nossa).}


% \section{Problemas de Otimização}

% \section{Problemas de Inovação}

% \section{Problemas de Aprendizagem Adaptativa}