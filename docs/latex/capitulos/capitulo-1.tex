\chapter{Instrodução}

Frente à complexidade das análises em economia, buscou-se continuamente desenvolver modelos, além melhorar os já existentes, que auxiliassem a encontrar respostas sobre os agentes econômicos e a sociedade como um todo. Ao longo do desenvolvimento das ciências econômicas como área de pesquisa, surgiram diversas teorias de como os agentes tomam suas decisões e, entendendo-as, como provê-las. Muitas destas possuem uma rígida base matemática que busca demonstrar sua robustez teórica e, dessa forma, dependem de uma grande quantidade de dados e variáveis como valores de entrada para os testes e simulações, o que pode demandar um alto custo computacional, além de certas limitações intrínsecas quando postas à prova em problemas complexos. Dessa forma, além da busca por modelos robustos visando um maior entendimento da conjectura econômica, tais modelos precisam se demonstrar confiáveis e eficientes quando aplicados em contextos com crescente complexidade. Com isso, buscou-se nesta monografia apresentar como os algoritmos genéticos podem ser uma opção alternativa ou complementar a modelos tradicionais 

Ademais, a economia se apresenta como uma área bastante interdisciplinar, fazendo interface com uma grande quantidade de campos de estudo, visando absorver conhecimentos e aprendizados em busca de entender o comportamento dos agentes econômicos. Além das áreas das ciências sociais e das ciências exatas que fizeram e fazem parte essencial na composição das teorias econômicas, diversas teorias das ciências biológicas também podem contribuir para o entendimento da sociedade sob a ótica econômica, seja olhando para o indivíduo ou para um conjunto de indivíduos. Dentre elas, está o estudo dos agentes econômicos sob a ótica evolucionária, sobretudo a teoria da evolução darwiniana, onde a capacidade de adaptação destes agentes em relação ao ambiente ditará sua sobrevivência em um dado contexto. Tais agentes, não se restringem, necessariamente, a pessoas. Qualquer objeto que compõe a estrutura de uma sociedade, seja uma organização, empresa, nação, etc., está passível de ser analisado através das teorias e modelos econômicos. Assim, com sua origem na biologia, os algoritmos genéticos demonstraram ser uma ferramenta robusta quando aplicados a diversos problemas de alta complexidade envolvendo o comportamento dos agentes.

Dessa forma, o presente trabalho tem como finalidade: i) realizar uma breve revisão histórica do contexto de criação e desenvolvimento dos algoritmos genéticos e da computação evolucionária; ii) aprofundar na estrutura de um algoritmo genético simples, apresentando os principais elementos para sua construção, como a formação da população inicial, os operadores de reprodução, cruzamento e mutação, a construção e definição da função objetivo e a teoria dos blocos de construção; iii) apresentar as teorias macroeconômicas com maior destaque no século XX e XXI e como os algoritmos genéticos podem ser aplicados a contextos econômicos e contribuírem para um maior entendimento dos indivíduos e da sociedade; iv) demonstrar, matemática e programaticamente, uma aplicação detalhada de um algoritmo genético simples na linguagem Python contendo os principais elementos para sua construção.

Para isso, como principais referências na construção e implementação de algoritmos genéticos, utilizou-se o artigo \enquote{\textit{Genetic Algorithms}} de John H. Holland, a obra \enquote{\textit{Genetic Algorithms in Search, Optimization and Machine Learning}} de David E. Goldberg e o livro \enquote{\textit{An Introduction to Genetic Algorithms}} de Melanie Mitchell. No que diz respeito à revisão história da computação evolucionária e dos algoritmos genéticos, foi utilizado o livro \enquote{\text{Handbook of Evolutionary Computation}} de Thomas Back, David B. Fogel e Zbigniew Michalewicz. Para definição de demais conceitos na área da computação e economia, foi utilizada uma compilação da bibliografia relevante ao tema como livros, teses, artigos acadêmicos e, no caso do código programático, as documentações necessárias.

Enquanto o capítulo dois traz uma revisão histórica da evolução do desenvolvimento da computação evolucionária, e dos algoritmos genéticos como sua sub área de pesquisa, o capitulo três faz um aprofundamento nos componentes para construção de um algoritmo genético contendo os operadores de reprodução, cruzamento e mutação, assim como a teoria por trás destes elementos. Já o capítulo quatro realiza uma breve revisão das principais teorias econômicas do século XX e XXI, uma apresentação da economia evolucionária e algumas aplicações de algoritmos genéticos em contextos econômicos, especialmente, em problemas de otimização, inovação e aprendizado adaptativo. No capítulo cinco, por sua vez, é feita, passo-a-passo, uma aplicação de um algoritmo genético simples demonstrando-a matematicamente, conforme exposto no capítulo três, assim como sua implementação para a resolução do Problema da Caixa Preta na linguagem Python. Por fim, apresenta-se a conclusão.
