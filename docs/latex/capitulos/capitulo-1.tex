\chapter{Introdução}

Frente à complexidade das análises em Economia, buscou-se continuamente ao longo do tempo desenvolver modelos, além de melhorar os já existentes, que auxiliassem a entender o comportamento dos agentes econômicos em busca de alcançar os objetivos definidos pela sociedade, sejam eles para atingir progresso tecnológico, aumentar o bem-estar social, elevar a felicidade dos indivíduos, entre outros. Ao longo do desenvolvimento das Ciências Econômicas como área de estudo, surgiram diversas teorias de como os agentes se comportam e tomam suas decisões quando expostos a alguns cenários específicos. Entretanto, apenas a fundamentação teórica não basta para a análise e o entendimento destes processos, mostrando-se necessária a construção de modelos matemáticos que visam demonstrar a robustez teórica dos argumentos apresentados. Os métodos, para isso, são diversos, porém podem ser divididos em dois caminhos gerais: modelos aplicados a dados reais e modelos aplicados a dados hipotéticos ou simulados. O caminho a ser seguindo dependerá do objetivo da pesquisa, das hipóteses a serem testadas, da disponibilidade de dados, de restrições computacionais, etc. Dessa forma, na busca por modelos robustos visando uma maior compreensão dos fenômenos econômicos, surgem diversas barreiras como a dificuldade em escalá-lo, a falta de dados estruturados e confiáveis para aplicação e treinamento, a necessidade de inserção de muitos parâmetros de entrada, a inflexibilidade frete à problemas complexos, entre muitas dificuldades encontradas quando postos à prova. Com isso, buscou-se, no presente trabalho, apresentar como os Algoritmos Genéticos podem ser uma alternativa robusta e eficiente para a compreensão e previsão dos comportamentos dos agentes em ambientes econômicos complexos, assim como contribuir para a literatura da área que carece de poucos trabalhos em língua portuguesa.

%Ademais, a Economia se apresenta como uma área bastante interdisciplinar, desenvolvendo ao longo do tempo interfaces sólidas com as demais áreas do conhecimento. Além das áreas das ciências sociais e das ciências exatas que fizeram e fazem parte essencial na composição das teorias econômicas, diversos fundamentos %das ciências biológicas foram absorvidos e contribuíram para a análise dos agentes sobre a ótica econômica, seja observando o comportamento de um indivíduo ou de toda a sociedade como, por exemplo, a teoria da evolução darwiniana, onde a capacidade de adaptação, ou sobrevivência, de um indivíduo, ou um grupo de %indivíduos, em relação ao ambiente, ditará a sua propagação para as próximas gerações. Os processos observados em ecossistemas naturais demonstram-se muito similares a algumas dinâmicas encontradas na economia como os processos de inovação tecnológica e o comportamento adaptativo dos agentes, apenas para citar %alguns. Assim, com sua origem na biologia, os Algoritmos Genéticos se demonstraram um método extremamente flexível, eficiente e de fácil explicabilidade e modelagem. Não foi diferente relação às aplicações em um grande número de problemas de alta complexidade envolvendo o comportamento dos agentes em uma economia.

Dessa forma, o presente trabalho tem como finalidade: i) realizar uma breve revisão histórica do contexto de criação e desenvolvimento dos Algoritmos Genéticos; ii) aprofundar na estrutura de um Algoritmo Genético simples, apresentando os principais elementos para sua construção, como a formação da população inicial, os operadores de reprodução, cruzamento e mutação, a construção e definição da função objetivo e a teoria dos blocos de construção; iii) apresentar exemplos de aplicações dos Algoritmos Genéticos em contextos econômicos como, por exemplo, problemas de otimização, análise de processos de inovação, análise de comportamento dos agentes, entre outros; iv) demonstrar, matemática e programaticamente, uma aplicação detalhada de um Algoritmo Genético simples na linguagem Python contendo os principais elementos para sua construção.

Para isso, como principais referências na construção, implementação e estrutura dos Algoritmos Genéticos, utilizou-se o artigo \enquote{\textit{Genetic Algorithms}} de John H. Holland, o livro \enquote{\textit{Genetic Algorithms in Search, Optimization and Machine Learning}} de David E. Goldberg e o livro \enquote{\textit{An Introduction to Genetic Algorithms}} de Melanie Mitchell. No que diz respeito à revisão história do desenvolvimento dos Algoritmos Genéticos, além do trabalho de Goldberg citado previamente, foi utilizado o livro \enquote{\text{Handbook of Evolutionary Computation}} de Thomas Back, David B. Fogel e Zbigniew Michalewicz. Para a apresentação dos exemplos de pesquisas em Algoritmos Genéticos aplicados a problemas econômicos, foi realizada uma revisão bibliográfica  do tema, com o aprofundamento em algumas das pesquisas encontradas na literatura. Para os demais conceitos e definições na área da Computação e Economia, foi utilizada uma compilação da bibliografia relevante às áreas como livros, teses, artigos acadêmicos e, no caso do código programático, as documentações necessárias.

Enquanto o capítulo dois traz uma revisão histórica da evolução do desenvolvimento da computação evolucionária, e dos algoritmos genéticos como sua sub área de pesquisa, o capitulo três faz um aprofundamento nos componentes para construção de um algoritmo genético contendo os operadores de reprodução, cruzamento e mutação, assim como a teoria por trás destes elementos. Já o capítulo quatro realiza uma breve revisão das principais teorias econômicas do século XX e XXI, uma apresentação da economia evolucionária e algumas aplicações de algoritmos genéticos em contextos econômicos, especialmente, em problemas de otimização, inovação e aprendizado adaptativo. No capítulo cinco, por sua vez, é feita, passo-a-passo, uma aplicação de um algoritmo genético simples demonstrando-a matematicamente, conforme exposto no capítulo três, assim como sua implementação para a resolução do Problema da Caixa Preta na linguagem Python. Por fim, apresenta-se a conclusão.
