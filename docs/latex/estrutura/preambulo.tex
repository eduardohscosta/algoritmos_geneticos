\documentclass[
	a4paper,
	12pt,
	openright,
	oneside,
	subsubsection=Title
	]{book}

\usepackage[english,brazil]{babel}
\usepackage{indentfirst}
\usepackage[utf8]{inputenc}
\usepackage{graphicx,color}
\usepackage{appendix}
\usepackage{longtable}
\usepackage{subfig}
\usepackage{float}
\usepackage{listings}
\usepackage{color}
\usepackage{indentfirst}
\usepackage{array}
\usepackage{csquotes}
\usepackage{xcolor}               %Extend color definitions
\usepackage{amsfonts}             %Use fonts   from AMS - American Mathematical Society 
\usepackage{amssymb}              %Use symbols from AMS - American Mathematical Society
\usepackage{amsmath}
\usepackage[authoryear]{natbib}   %Use package to bibliography - author(year)
\usepackage{makeidx}
\usepackage{supertabular}         %permite tabela continuar em sucessivas p�ginas 
\usepackage{marvosym}             %Simbolos diversos
\usepackage{verbatim} 
\usepackage{listings}
\usepackage{multicol}
\usepackage[subfigure]{tocloft}
\usepackage{multirow}
\usepackage{titlesec}
\usepackage[
	pdftex,
	colorlinks,
	linkcolor=blue, 
	citecolor=red!50!black,
	hyperindex
	]{hyperref} %Permite fazer hypertexto


% Alteração da indentação dos itens do sumário
\cftsetindents{chapter}{0pt}{42pt}
\cftsetindents{section}{0pt}{42pt}
\cftsetindents{subsection}{0pt}{42pt}
\cftsetindents{subsubsection}{0pt}{42pt}

\hypersetup{pdfstartview=FitH}

\setcounter{secnumdepth}{3}

\lstset{morekeywords=[1]{to,to-report,end, extensions,to-report,globals,breed,directed-link-breed, link-breed},
	extendedchars=true, 
	breaklines=true,
	breakatwhitespace=true,
	basicstyle=\ttfamily\scriptsize,
	columns=fullflexible,
	tabsize=4,
	keywordstyle=[1]\color[rgb]{0.25,0.5,0.35},%\color[rgb]{0.4,0.8,0.2}\bfseries,
	morekeywords=[2]{let,set,loop, if, report, foreach, print, tick, while, every, ifelse, ask},
	keywordstyle=[2]\color{blue},
	alsoletter={-,?,.},
	morekeywords=[3]{position, not, reverse, fput, substring, length, word, timer,is-number?,filter,first, bf, butfirst, last, empty?, n-of, color, who, shape},
	keywordstyle=[3]\color[rgb]{0.6,0,0.8},
	comment=[l]{\;},
	commentstyle=\color[rgb]{0.75,0.75,0.75},
	string=[d]{"},
	stringstyle=\color{orange},
	otherkeywords={1, 2, 3, 4, 5, 6, 7, 8, 9, 0},
	morekeywords=[4]{1, 2, 3, 4, 5, 6, 7, 8, 9, 0, true, false},
	keywordstyle=[4]{\color{orange}}}

% Default fixed font does not support bold face
\DeclareFixedFont{\ttb}{T1}{txtt}{bx}{n}{12} % for bold
\DeclareFixedFont{\ttm}{T1}{txtt}{m}{n}{12}  % for normal

% Custom colors
\usepackage{color}
\definecolor{deepblue}{rgb}{0,0,0.5}
\definecolor{deepred}{rgb}{0.6,0,0}
\definecolor{deepgreen}{rgb}{0,0.5,0}

\usepackage{listings}

% https://tex.stackexchange.com/questions/83882/how-to-highlight-python-syntax-in-latex-listings-lstinputlistings-command()
% https://nasa.github.io/nasa-latex-docs/html/examples/listing.html
% Python style for highlighting
\newcommand\pythonstyle{\lstset{
		language=Python,
		basicstyle=\footnotesize, 
		keywordstyle=\footnotesize\color{deepblue},
		emphstyle=\footnotesize\color{deepred},
		commentstyle=\footnotesize\color{deepgreen},
		stringstyle=\footnotesize\color{deepgreen},
		showstringspaces=false,
		breaklines=true,
}}

% Python environment
\lstnewenvironment{python}[1][]
{
	\pythonstyle
	\lstset{#1}
}
{}

% Python for external files
\newcommand\pythonexternal[2][]{{
		\pythonstyle
		\lstinputlisting[#1]{#2}}}

% Python for inline
\newcommand\pythoninline[1]{{\pythonstyle\lstinline!#1!}}

%Defini��es de Margens
\addtolength{\textwidth}{2.0cm}
\addtolength{\textheight}{2.0cm}
\addtolength{\voffset}{-0.5cm}
\addtolength{\hoffset}{-0.5cm}

\title{{\color{black}\textbf{Algoritmo Genético Aplicado a Problemas na Economia}}}
\author{\Large\textbf{Eduardo Henrique Silveira da Costa}\\
	Departamento de Economia\\ 
	Universidade Federal do Paraná - UFPR \\
	Curitiba - Paraná\\
	Brasil\\ \\ \\ \\
	\textbf{eduardohsdacosta@gmail.com} \\ \\ \\ \\ \\ \\ \\ \\}
\pagestyle{plain}
\includeonly{%
	./capitulos/capitulo-1,
	./capitulos/capitulo-2,
	./capitulos/capitulo-3,
	./capitulos/capitulo-4,
	./capitulos/capitulo-5,
	./capitulos/capitulo-6
}