% ----------------------------------------------------------------------------------------------------
% ----------------------------------------------------------------------------------------------------
% Projeto Latex de Monografia apresentada à Universidade Federal do Paraná (UFPR)
% Distribuição Tex: MikTeX (https://miktex.org);
% Construção/Edição: TeXstudio (https://www.texstudio.org);
% Compilação: Texstudio/Texworks (https://www.texstudio.org; https://tug.org/texworks/);
% ---------------------------------------------------------------------------------------------------
% ---------------------------------------------------------------------------------------------------

% %%%%%%%%%%%%%%%%%%%%%%%%%%%%%%%%%%%%%%%%%%%%%%%%%%%%%%%%%%%%%%%%%%%%%%%%%%%%%%%%%%%%%%%%%%%%%%%%%%%
% ALGORITMO GENÉTICO APLICADO A PROBLEMAS NA ECONOMIA
% %%%%%%%%%%%%%%%%%%%%%%%%%%%%%%%%%%%%%%%%%%%%%%%%%%%%%%%%%%%%%%%%%%%%%%%%%%%%%%%%%%%%%%%%%%%%%%%%%%%

\documentclass[
	a4paper,
	12pt,
	openright,
	oneside,
	subsubsection=Title
	]{book}

\usepackage[english,brazil]{babel}
\usepackage{indentfirst}
\usepackage[utf8]{inputenc}
\usepackage{graphicx,color}
\usepackage{appendix}
\usepackage{longtable}
\usepackage{subfig}
\usepackage{float}
\usepackage{listings}
\usepackage{color}
\usepackage{indentfirst}
\usepackage{array}
\usepackage{csquotes}
\usepackage{xcolor}               %Extend color definitions
\usepackage{amsfonts}             %Use fonts   from AMS - American Mathematical Society 
\usepackage{amssymb}              %Use symbols from AMS - American Mathematical Society
\usepackage{amsmath}
\usepackage[authoryear]{natbib}   %Use package to bibliography - author(year)
\usepackage{makeidx}
\usepackage{supertabular}         %permite tabela continuar em sucessivas p�ginas 
\usepackage{marvosym}             %Simbolos diversos
\usepackage{verbatim} 
\usepackage{listings}
\usepackage{multicol}
\usepackage[subfigure]{tocloft}
\usepackage{multirow}
\usepackage{titlesec}
\usepackage[
	pdftex,
	colorlinks,
	linkcolor=blue, 
	citecolor=red!50!black,
	hyperindex
	]{hyperref} %Permite fazer hypertexto


% Alteração da indentação dos itens do sumário
\cftsetindents{chapter}{0pt}{42pt}
\cftsetindents{section}{0pt}{42pt}
\cftsetindents{subsection}{0pt}{42pt}
\cftsetindents{subsubsection}{0pt}{42pt}

\hypersetup{pdfstartview=FitH}

\setcounter{secnumdepth}{3}

\lstset{morekeywords=[1]{to,to-report,end, extensions,to-report,globals,breed,directed-link-breed, link-breed},
	extendedchars=true, 
	breaklines=true,
	breakatwhitespace=true,
	basicstyle=\ttfamily\scriptsize,
	columns=fullflexible,
	tabsize=4,
	keywordstyle=[1]\color[rgb]{0.25,0.5,0.35},%\color[rgb]{0.4,0.8,0.2}\bfseries,
	morekeywords=[2]{let,set,loop, if, report, foreach, print, tick, while, every, ifelse, ask},
	keywordstyle=[2]\color{blue},
	alsoletter={-,?,.},
	morekeywords=[3]{position, not, reverse, fput, substring, length, word, timer,is-number?,filter,first, bf, butfirst, last, empty?, n-of, color, who, shape},
	keywordstyle=[3]\color[rgb]{0.6,0,0.8},
	comment=[l]{\;},
	commentstyle=\color[rgb]{0.75,0.75,0.75},
	string=[d]{"},
	stringstyle=\color{orange},
	otherkeywords={1, 2, 3, 4, 5, 6, 7, 8, 9, 0},
	morekeywords=[4]{1, 2, 3, 4, 5, 6, 7, 8, 9, 0, true, false},
	keywordstyle=[4]{\color{orange}}}

% Default fixed font does not support bold face
\DeclareFixedFont{\ttb}{T1}{txtt}{bx}{n}{12} % for bold
\DeclareFixedFont{\ttm}{T1}{txtt}{m}{n}{12}  % for normal

% Custom colors
\usepackage{color}
\definecolor{deepblue}{rgb}{0,0,0.5}
\definecolor{deepred}{rgb}{0.6,0,0}
\definecolor{deepgreen}{rgb}{0,0.5,0}

\usepackage{listings}

% https://tex.stackexchange.com/questions/83882/how-to-highlight-python-syntax-in-latex-listings-lstinputlistings-command()
% https://nasa.github.io/nasa-latex-docs/html/examples/listing.html
% Python style for highlighting
\newcommand\pythonstyle{\lstset{
		language=Python,
		basicstyle=\footnotesize, 
		keywordstyle=\footnotesize\color{deepblue},
		emphstyle=\footnotesize\color{deepred},
		commentstyle=\footnotesize\color{deepgreen},
		stringstyle=\footnotesize\color{deepgreen},
		showstringspaces=false,
		breaklines=true,
}}

% Python environment
\lstnewenvironment{python}[1][]
{
	\pythonstyle
	\lstset{#1}
}
{}

% Python for external files
\newcommand\pythonexternal[2][]{{
		\pythonstyle
		\lstinputlisting[#1]{#2}}}

% Python for inline
\newcommand\pythoninline[1]{{\pythonstyle\lstinline!#1!}}

%Defini��es de Margens
\addtolength{\textwidth}{2.0cm}
\addtolength{\textheight}{2.0cm}
\addtolength{\voffset}{-0.5cm}
\addtolength{\hoffset}{-0.5cm}

\title{{\color{black}\textbf{Algoritmo Genético Aplicado a Problemas na Economia}}}
\author{\Large\textbf{Eduardo Henrique Silveira da Costa}\\
	Departamento de Economia\\ 
	Universidade Federal do Paraná - UFPR \\
	Curitiba - Paraná\\
	Brasil\\ \\ \\ \\
	\textbf{eduardohsdacosta@gmail.com} \\ \\ \\ \\ \\ \\ \\ \\}
\pagestyle{plain}
\includeonly{%
	./capitulos/capitulo-1,
	./capitulos/capitulo-2,
	./capitulos/capitulo-3,
	./capitulos/capitulo-4,
	./capitulos/capitulo-5,
	./capitulos/capitulo-6
}
% Impressão da Capa
%----------------------------------------------------------------------------- 
\newcommand{\capa}[5]{%
    \thispagestyle{empty}
    \begin{center}
    
        % Universidade
        \vfill
        \MakeUppercase{#1}
        
        % Autor
        \vspace{4cm}
        \MakeUppercase{#2}
        
        % Nome Monografia
        \vspace{4cm}
        \MakeUppercase{#3}
        
        % Cidade
        \vspace{12cm}
        \MakeUppercase{#4}
        
        % Ano
        {#5}
    \end{center}
}

% Impressão da Folha de Rosto
%----------------------------------------------------------------------------- 
\newcommand{\folharosto}[6]{%
	\thispagestyle{empty}
	\pagenumbering{gobble}
	\begin{center}
	        
		% Autor
		\vspace{4cm}
		\MakeUppercase{#1}	
	        
		% Nome Monografia
		\vspace{4cm}
		\MakeUppercase{#2} \linebreak[2]
	
		% Orientador
		\hspace{6cm}
		\begin{minipage}{.5\textwidth}
			\small
			\scriptsize
			{#3}
			\\
			\\
			{#4}
		\end{minipage}
	
		% Cidade
		\vspace{14cm}
		\MakeUppercase{#5}
	        
		% Ano
		{#6}
		\vfill
	\end{center}
}

% Impressão da Folha do Termo de aprovação
%----------------------------------------------------------------------------- 
\newcommand{\folhatermoaprovacao}[8]{%
	\newpage
	\thispagestyle{empty}
	\pagenumbering{gobble}
	\begin{center}
		\uppercase{\textbf{Termo de Aprovação}}
	
		% Nome Autor
		\vspace{2cm}
		\uppercase{#1}
	
		% Nome Monogfaria
		\vspace{2cm}
		\uppercase{#2}
	\end{center}

	% Informações sobre a monofrafia
	\vspace{2cm}
	\noindent {#3}

	\vspace{2cm}
	\begin{tabular}{l}
		\hline \\
		\small
		% Professor orientador
		Orientador: {#4} \\
		Setor de Ciências Sociais Aplicadas \\
		Universidade Federal do Paraná   \\
		\vspace{25pt}  \\
		\hline \\
		\small
		% Professor 1 da banca
		{#5} \\
		Setor de Ciências Sociais Aplicadas \\
		Universidade Federal do Paraná   \\
		\vspace{25pt}   \\
		\hline \\
		\small
		% Professor 2 da banca
		{#6} \\
		Setor de Ciências Sociais Aplicadas \\
		Universidade Federal do Paraná   \\
		\vspace{25pt}   \\
	\end{tabular}

\begin{center}
	\small
	% {Dia} de {Mês} de {Ano}
	{#7}, {#8}.
\end{center}
}

% Impressão da Folha de Agradecimentos
%----------------------------------------------------------------------------- 
\newcommand{\agradecimentos}[1]{%
	\newpage
	\thispagestyle{empty}
	\pagenumbering{gobble}
	\begin{center}
		\uppercase{\textbf{Agradecimentos}}
	\end{center}

	% Agradecimentos
	\vspace{2cm}
	#1
}

% Impressão da Folha de Resumo/Abstract
%----------------------------------------------------------------------------- 
\newcommand{\resumo}[4]{%
	\newpage
	\thispagestyle{empty}
	\pagenumbering{gobble}

	\begin{center}
		{\large \uppercase{\textbf{Resumo}}}
	\end{center}

	% Resumo
	\vspace{0.5cm}
	\noindent
	{#1} 
	
	% Palavras-Chave
	\vspace{0.5cm}
	\noindent Palavras-chave: {#2}
	
	\vspace{2cm}
	
	\begin{center}
		{\large \uppercase{\textbf{Abstract}}}
	\end{center}

	% Abstract
	\vspace{0.5cm}
	\noindent
	{#3}
	
	% Keywords
	\vspace{0.5cm}
	\noindent Keywords: {#4}
}

% Impressão da Folha de Lista de Figuras
%----------------------------------------------------------------------------- 
\newcommand{\listafiguras}{
	\newpage
	\thispagestyle{empty}
	\pagenumbering{gobble}
	\listoffigures
}

% Impressão da Folha de Lista de Tebelas
%----------------------------------------------------------------------------- 
\newcommand{\listatabelas}{
	\newpage
	\thispagestyle{empty}
	\pagenumbering{gobble}
	\listoftables
}

% Impressão da Folha do Sumário
%----------------------------------------------------------------------------- 
\newcommand{\sumario}{
	\newpage
	\thispagestyle{empty}
	\pagenumbering{gobble}
	\tableofcontents
}

% Impressão da Folha de Siglas
%-----------------------------------------------------------------------------
\newcommand{\listasiglas}[1]{%
	\newpage
	\thispagestyle{empty}
	\pagenumbering{gobble}
	
	\begin{center}
		{\large \uppercase{\textbf{Lista de Siglas}}}
	\end{center}
	
	% Siglas
	\vspace{0.5cm}
	\noindent
	{#1}
}
% ----------------------------------------------------------------------------------------------------
% ----------------------------------------------------------------------------------------------------
% All new command documantation can be found in utils/comandos_doc.tex
% ----------------------------------------------------------------------------------------------------
% ----------------------------------------------------------------------------------------------------

% Multiline comment in Latex script. 
% ----------------------------------------------------------------------------------------------------
\newcommand{\multilinecomment}[1]{}

% Table macro insertion.
% ----------------------------------------------------------
\newcommand{\tabelasimples}[5]{%
	\begin{table}[H]
		\centering
		\small
		\def\arraystretch{1.2}
        \caption{\centering\uppercase{#2}}
        \vspace{2pt}
        \begin{tabular}[c]{#1}
        	\hline
            #3 %
        \end{tabular}
        \begin{minipage}{1\textwidth}
            \vspace{6pt}
            \centering
            \par FONTE:~#5
            \label{tab:#4}
        \end{minipage}
    \end{table}
}

% Table macro insertion.
% ----------------------------------------------------------
\newcommand{\tabelamultilinhas}[5]{%
	\begin{table}[H]
		\centering
		\small
		\def\arraystretch{1.2}
		\caption{\centering\uppercase{#2}}
		\vspace{2pt}
		\begin{tabular}[c]{#1}
			#3 %
		\end{tabular}
		\begin{minipage}{1\textwidth}
			\vspace{6pt}
			\centering
			\par FONTE:~#5
			\label{tab:#4}
		\end{minipage}
	\end{table}
}

% Table macro insertion.
% ----------------------------------------------------------
\newcommand{\tabelamulticolunas}[5]{%
	\begin{table}[H]
		\centering
		\small
		\def\arraystretch{1.2}
		\caption{\centering\uppercase{#2}}
		\vspace{2pt}
		\begin{tabular}[c]{#1}
			#3 %
		\end{tabular}
		\begin{minipage}{1\textwidth}
			\vspace{6pt}
			\centering
			\par FONTE:~#5
			\label{tab:#4}
		\end{minipage}
	\end{table}
}

% Table macro insertion.
% ----------------------------------------------------------
\newcommand{\tabela}[4]{%
	\begin{table}[H]
		\centering
		\small
		\def\arraystretch{1.2}
		\caption{\centering\uppercase{#1}}
		\vspace{2pt}
		{#2}
		\begin{minipage}{1\textwidth}
			\vspace{6pt}
			\centering
			\par FONTE:~#4
			\label{tab:#3}
		\end{minipage}
	\end{table}
}

% Table macro insertion.
% ----------------------------------------------------------
\newcommand{\tabelalonga}[3]{%
	\begin{longtable}{#1} \hline
		\centering
		\small
		\def\arraystretch{2}
		#2
	\end{longtable}
}

% Figure insertion macro
% ----------------------------------------------------------

\newcommand{\figura}[5]{%
    \begin{center}
    	\small
        \begin{figure}[H] % Places the float at precisely the location in the LATEX code
            \centering
                \uppercase{\caption{\centering\uppercase{#1}}} % Define figure title
                \includegraphics[width=#2\textwidth]{#3} % Define figure proportion and the figure path
            %ajustado p/ a largura da imagem
            \begin{minipage}{#2\textwidth} 
                \vspace{0mm}
                \centering
                \par FONTE: ~#5 % Define the figure source
                \label{fig:#4}
            \end{minipage}
        \end{figure}
    \end{center}
}

\newcommand{\figuracomnota}[6]{%
	\begin{center}
		\small
		\begin{figure}[H] % Places the float at precisely the location in the LATEX code
			\centering
			\uppercase{\caption{\centering\uppercase{#1}}} % Define figure title
			\includegraphics[width=#2\textwidth]{#3} % Define figure proportion and the figure path
			%ajustado p/ a largura da imagem
			\begin{minipage}{#2\textwidth}
				\vspace{0mm}
				{\centering FONTE: ~#5 \par } % Define the figure source
				\hspace{20pt} NOTA: ~#6 \par % Define the figure extra information
			\end{minipage}
		\label{fig:#4}
		\end{figure}
	\end{center}
}

% Citação
% ----------------------------------------------------------
\newcommand{\citacao}[4]{%
	\vspace{0.5cm}
	\hspace{4cm}
	\begin{minipage}{0.65\textwidth}
		\scriptsize
		\setlength{\parindent}{0em}
		#1~#3~#4
	\end{minipage}
	\footnote{\textit{#2}}
	\vspace{0.5cm}
}

% Citação apud
% ----------------------------------------------------------
\newcommand{\citeapud}[2]{%
	(\Citealp{#1} apud \Citealp{#2})
}

\newcommand{\citeapudyear}[2]{%
	(\citeyear{#1} apud \Citealp{#2})
}

\renewcommand{\footnotesize}{%
	\fontsize{9pt}{10pt}\selectfont
}


\captionsetup{figurename=FIGURA}
\captionsetup[table]{name=TABELA}

% ---------------------------------------------------------------------------------------------------
% Início do Documento
% ---------------------------------------------------------------------------------------------------

\begin{document}

%Estilo deve ser compativel com o pacote NATBIB [author(year) citation] - Usa um arquivo xxxx.bst
\bibliographystyle{apa} 
\renewcommand{\bibname}{Referências}
\renewcommand{\contentsname}{\centering \large \uppercase{Sumário}}
\renewcommand{\listfigurename}{\centering \large \uppercase{Lista de Figuras}}
\renewcommand{\listtablename}{\centering \large \uppercase{Lista de Tabelas}}
\renewcommand{\abstractname}{Resumo}
\renewcommand\appendixname{Apêndice}
\renewcommand\appendixpagename{Apêndices}
	
% ---------------------------------------------------------------------------------------------------
% Construção da Capa
% ---------------------------------------------------------------------------------------------------

\capa
	{Universidade Federal do Paraná}
	{Eduardo Henrique Silveira da Costa}
	{Algoritmo Genético Aplicado a Problemas na Economia}
	{Curitiba}
	{2022}

% ---------------------------------------------------------------------------------------------------
% Construção da Folha de Rosto
% ---------------------------------------------------------------------------------------------------

\folharosto
	{Eduardo Henrique Silveira da Costa}
	{Algoritmo Genético Aplicado a Problemas na Economia}
	{Monografia apresentada como requisito à obtenção do grau de Bacharel em Ciências Econômicas pelo Curso de Ciências Econômicas, Setor de Ciências Sociais Aplicada, Universidade Federal do Paraná.}
	{Orientador: Dr. João Basilio Pereima}
	{Curitiba}
	{2022}

% ---------------------------------------------------------------------------------------------------
% Construção da Folha com Termos de Aprovação
% ---------------------------------------------------------------------------------------------------

\folhatermoaprovacao
	{Eduardo Henrique Silveira da Costa}
	{Algoritmo Genético Aplicado a Problemas na Economia}
	{Monografia apresentada como requisito à obtenção do grau de Bacharel em Ciências Econômicas pelo Curso de Ciências Econômicas, Setor de Ciências Sociais Aplicada, Universidade Federal do Paraná:}
	{Prof. Dr. João Basilio Pereima}
	{Profa. Dra. Adriana Sbicca}
	{Prof. Dr. José Guilherme Vieira}
	{Curitiba}
	{25 de Abril de 2022}

% ---------------------------------------------------------------------------------------------------
% Construção da Folha de Agradecimentos
% ---------------------------------------------------------------------------------------------------
%\agradecimentos
%	{XXXXXXX}

% ---------------------------------------------------------------------------------------------------
% Construção da Folha Resumo e Abstract
% ---------------------------------------------------------------------------------------------------
\resumo
	{%
		Em economia, embora haja algumas teorias bem estabelecidas, não há um consenso de como se analisar certos comportamentos dos agentes econômicos. Algumas linhas de pensamento se destacaram e tornaram-se a principal forma de análise e previsão de conjecturas econômicas ao redor do mundo em diferentes momentos do século XX e XXI. Paralelamente, a partir da metade do século passado, os sistemas computacionais se demonstram uma ferramenta essencial para desenvolvimento e testes de modelos. Muitos destes, por sua vez, apresentam limitações na busca por respostas e soluções de problemas econômicos complexos, seja por restrições teóricas ou computacionais. Dessa forma, a presente monografia tem o objetivo de apresentar os algoritmos genéticos como alternativa para a análise e previsão de cenários econômicos, especialmente, no que diz respeito a problemas de otimização, inovação e aprendizado adaptativo. Realizou-se uma breve apresentação da história de criação e desenvolvimento dos algoritmos genéticos, o aprofundamento em sua estrutura, aplicações em problemas econômicos e uma aplicação passo-a-passo de um algoritmo simples na linguagem Python.
	}
	{algoritmo genético; otimização; inovação; aprendizado adaptativo.}
	{%
		In economics, there are some theories well, there is no consensus on how to accept agent patterns. Some lines of thought stood out and became the main form of analysis and prediction of conjectures around the world at different times in the 20th and 21st. At the same time, from the middle of the last century, the computer systems of the century have been essential tools for model development and testing. Many of these, in turn, present solutions to complex problems by their answers, sometimes, proposals of complex problems, that is, sometimes, proposals, for their solution, in turn, proposals, for their solution. In this way, the monograph has the objective of innovation, the purpose of presenting genetic beings as the proposal of refuge and the prediction of environments, especially, the concern and prediction of environments, it does not concern optimization problems, especially, and adaptive learning. An evolution of the history of creation and development of animals was carried out, genetic deepening, applications in problems the development of a simple step-by-step application of a simple environment in its structure.
	}
	{genetic algorithm; optimization; inovation; adaptative learning.}

% ---------------------------------------------------------------------------------------------------
% Construção da Lista de Figuras
% ---------------------------------------------------------------------------------------------------
\listafiguras
 
% ---------------------------------------------------------------------------------------------------
% Construção da Lista de Tabelas
% ---------------------------------------------------------------------------------------------------
\listatabelas
 
% ---------------------------------------------------------------------------------------------------
% Construção da Lista de Siglas
% ---------------------------------------------------------------------------------------------------
\listasiglas
	{%
		AG - Algoritmo Genético \\
		CE - Computação Evolucionária \\
		PE - Programação Evolucionária \\
		EE - Estratégias Evolutivas \\
		PIB - Produto Interno Bruto
	}

% ---------------------------------------------------------------------------------------------------
% Construção do Sumário
% ---------------------------------------------------------------------------------------------------
\sumario

%-------------------------------------------------------------------------------------
\pagenumbering{arabic}
\newpage
\chapter{Introdução}

Frente à complexidade das análises em Economia, buscou-se continuamente ao longo do tempo desenvolver modelos, além de melhorar os já existentes, que auxiliassem a entender o comportamento dos agentes econômicos em busca de alcançar os objetivos definidos pela sociedade, sejam eles para atingir progresso tecnológico, aumentar o bem-estar social, elevar a felicidade dos indivíduos, entre outros. Ao longo do desenvolvimento das Ciências Econômicas como área de estudo, surgiram diversas teorias de como os agentes se comportam e tomam suas decisões quando expostos a alguns cenários específicos. Entretanto, apenas a fundamentação teórica não basta para a análise e o entendimento destes processos, mostrando-se necessária a construção de modelos matemáticos que visam demonstrar a robustez teórica dos argumentos apresentados. Os métodos, para isso, são diversos, porém podem ser divididos em dois caminhos gerais: modelos aplicados a dados reais e modelos aplicados a dados hipotéticos ou simulados. O caminho a ser seguindo dependerá do objetivo da pesquisa, das hipóteses a serem testadas, da disponibilidade de dados, de restrições computacionais, etc. Dessa forma, na busca por modelos robustos visando uma maior compreensão dos fenômenos econômicos, surgem diversas barreiras como a dificuldade em escalá-lo, a falta de dados estruturados e confiáveis para aplicação e treinamento, a necessidade de inserção de muitos parâmetros de entrada, a inflexibilidade frete à problemas complexos, entre muitas dificuldades encontradas quando postos à prova. Com isso, buscou-se, no presente trabalho, apresentar como os Algoritmos Genéticos podem ser uma alternativa robusta e eficiente para a compreensão e previsão dos comportamentos dos agentes em ambientes econômicos complexos, assim como contribuir para a literatura da área que carece de poucos trabalhos em língua portuguesa.

%Ademais, a Economia se apresenta como uma área bastante interdisciplinar, desenvolvendo ao longo do tempo interfaces sólidas com as demais áreas do conhecimento. Além das áreas das ciências sociais e das ciências exatas que fizeram e fazem parte essencial na composição das teorias econômicas, diversos fundamentos %das ciências biológicas foram absorvidos e contribuíram para a análise dos agentes sobre a ótica econômica, seja observando o comportamento de um indivíduo ou de toda a sociedade como, por exemplo, a teoria da evolução darwiniana, onde a capacidade de adaptação, ou sobrevivência, de um indivíduo, ou um grupo de %indivíduos, em relação ao ambiente, ditará a sua propagação para as próximas gerações. Os processos observados em ecossistemas naturais demonstram-se muito similares a algumas dinâmicas encontradas na economia como os processos de inovação tecnológica e o comportamento adaptativo dos agentes, apenas para citar %alguns. Assim, com sua origem na biologia, os Algoritmos Genéticos se demonstraram um método extremamente flexível, eficiente e de fácil explicabilidade e modelagem. Não foi diferente relação às aplicações em um grande número de problemas de alta complexidade envolvendo o comportamento dos agentes em uma economia.

Dessa forma, o presente trabalho tem como finalidade: i) realizar uma breve revisão histórica do contexto de criação e desenvolvimento dos Algoritmos Genéticos; ii) aprofundar na estrutura de um Algoritmo Genético simples, apresentando os principais elementos para sua construção, como a formação da população inicial, os operadores de reprodução, cruzamento e mutação, a construção e definição da função objetivo e a teoria dos blocos de construção; iii) apresentar exemplos de aplicações dos Algoritmos Genéticos em contextos econômicos como, por exemplo, problemas de otimização, análise de processos de inovação, análise de comportamento dos agentes, entre outros; iv) demonstrar, matemática e programaticamente, uma aplicação detalhada de um Algoritmo Genético simples na linguagem Python contendo os principais elementos para sua construção.

Para isso, como principais referências na construção, implementação e estrutura dos Algoritmos Genéticos, utilizou-se o artigo \enquote{\textit{Genetic Algorithms}} de John H. Holland, o livro \enquote{\textit{Genetic Algorithms in Search, Optimization and Machine Learning}} de David E. Goldberg e o livro \enquote{\textit{An Introduction to Genetic Algorithms}} de Melanie Mitchell. No que diz respeito à revisão história do desenvolvimento dos Algoritmos Genéticos, além do trabalho de Goldberg citado previamente, foi utilizado o livro \enquote{\text{Handbook of Evolutionary Computation}} de Thomas Back, David B. Fogel e Zbigniew Michalewicz. Para a apresentação dos exemplos de pesquisas em Algoritmos Genéticos aplicados a problemas econômicos, foi realizada uma revisão bibliográfica  do tema, com o aprofundamento em algumas das pesquisas encontradas na literatura. Para os demais conceitos e definições na área da Computação e Economia, foi utilizada uma compilação da bibliografia relevante às áreas como livros, teses, artigos acadêmicos e, no caso do código programático, as documentações necessárias.

Enquanto o capítulo dois traz uma revisão histórica da evolução do desenvolvimento da computação evolucionária, e dos algoritmos genéticos como sua sub área de pesquisa, o capitulo três faz um aprofundamento nos componentes para construção de um algoritmo genético contendo os operadores de reprodução, cruzamento e mutação, assim como a teoria por trás destes elementos. Já o capítulo quatro realiza uma breve revisão das principais teorias econômicas do século XX e XXI, uma apresentação da economia evolucionária e algumas aplicações de algoritmos genéticos em contextos econômicos, especialmente, em problemas de otimização, inovação e aprendizado adaptativo. No capítulo cinco, por sua vez, é feita, passo-a-passo, uma aplicação de um algoritmo genético simples demonstrando-a matematicamente, conforme exposto no capítulo três, assim como sua implementação para a resolução do Problema da Caixa Preta na linguagem Python. Por fim, apresenta-se a conclusão.
	% Instrodução
\chapter{Computação Evolucionária}

O contexto da criação e desenvolvimento dos algoritmos genéticos está intrinsecamente ligado à área de estudos da computação evolucionária que, por sua vez, está sob o guarda-chuva da área de pesquisa da inteligência artificial. Com o objetivo de entender o desenvolvimento dos algoritmos genéticos, desde sua criação, procura-se, neste capítulo, realizar uma contextualização histórica à computação evolucionária, à programação evolucionária, às estratégias evolutivas, além da evolução e principais contribuições para o desenvolvimento dos algoritmos genéticos até sua formalização como campo de pesquisa.

\section{Introdução}

Como o próprio nome sugere, a computação evolucionária (CE) tem como inspiração os processos de evolução observados nos organismos da natureza, sendo uma metáfora computacional (vide \autoref{tab:EvolutionaryComputingMetaphor}) que, de forma geral, visa solucionar problemas computacionais ou entender melhor os processos naturais de evolução. Através de uma simulação destes processos naturais, busca-se pelos indivíduos mais aptos a sobreviverem em um ambiente, assim como analisar como os processos de reprodução e mutação destes indivíduos ocorreram. O próprio ambiente é, em si, um dos elementos mais importantes neste conjunto, tendo grande influência nessa luta pela sobrevivência e na busca de parceiros para reprodução, determinando, assim, como a capacidade de se adaptar a esse meio influenciará em suas chances de passar seus genes para as próximas gerações.

\tabelasimples
	{ccc}
	{Metáfora da Computação Evolucionária}
	{%
		Evolution & ~ & Problem solving \\ \hline
		Environment & $\Longleftrightarrow$ & Problem \\
		Individual & $\Longleftrightarrow$ & Candidate solution \\
		Fitness & $\Longleftrightarrow$ & Quality \\ \hline
	}
	{EvolutionaryComputingMetaphor}
	{\citet[p.11]{eiben_introduction_2015}}

Ao fim de 1950, e meados da década de 60, a tecnologia havia avançado até chegar à computação digital, o que possibilitou um avanço na experimentação de novos modelos de processos evolucionários e um grande número de estudos nas décadas seguintes. Os trabalhos de \cite{learning_machine_1958}, \cite{Friedberg1959ALM} e \cite{bremermann_optimization_1962} são apontados como os primeiros registros de desenvolvimento de processos evolucionários aplicados no contexto de problemas computacionais. Os trabalhos de Friedberg podem ser considerados alguns dos primeiros estudos em \textit{machine learning}\footnote{Do inglês, aprendizado de máquina.} e programação automática \Citep{back_handbook_1997}.

\cite{bremermann_optimization_1962}, publica sua pesquisa de evolução simulada aplicada à otimização linear e convexa e equações simultâneas não lineares, assim como desenvolve, em 1965\footnote{\cite{search_evolution_biophysics_1965}}, um dos primeiros estudos teóricos sobre algoritmos evolucionários, demonstrando que a mutação ótima deve ter um valor $\frac{1}{l}$\footnote{\textit{length}, do inglês, comprimento} no caso de $l$ bits codificados como indivíduos quando aplicados a problemas linearmente separáveis \Citep{back_handbook_1997}.

Com as contribuições dos trabalhados acima apresentados, as pesquisas realizadas na segunda metade dos anos 1960 estabeleceram os três principais campos de estudo em CE, sendo eles a programação evolucionária (PE), as estratégias evolutivas (EE) e os algoritmos genéticos (AG). Na segunda metade da década de 1960, Lawrance Fogel\footnote{\cite{Fogel1966ArtificialIT}} construía as bases da programação evolucionária em San Diego, Califórnia, e John Holland fundava as bases dos algoritmos genéticos na Universidade de Michigan\footnote{\cite{holland_1962}}. Por sua vez, as estratégias evolutivas eram desenvolvidas por Inge Rechenberg, Peter Bienert e Hans-Paul Schwefel em Berlim em meados de 1965\footnote{\cite{rechenberg_cybernetic_1965}}.

Como aponta \cite{back_handbook_1997}, mesmo com cada uma das áreas seguindo seu próprio caminho de pesquisas ao longo dos quase 30 anos seguintes, a década de 1990 marca o encontro destes campos através dos esforços de seus pesquisadores na organização de diversos congressos com o objetivo de compartilharem os conhecimentos até então absorvidos, culminando, no início da década, no consenso do nome \textbf{computação evolucionária} como o nome dessa nova grande área de pesquisa. A partir destas reuniões, o crescimento do número de interessados e novos trabalhos foi naturalmente crescendo ao longo da década. Em 1993, é criado um periódico homônimo pelo Instituto de Tecnologia de Massachusetts\footnote{Evolutionary Computation \citeyearpar{noauthor_editorial_1993}} e, em 1994, uma das três conferências do Congresso Mundial de Inteligência Computacional organizado pelo Instituto de Engenheiros Eletricistas e Eletrônicos\footnote{\textit{IEEE World Congress on Computaional Intelligence} (WCCI)} \citeapud{back_handbook_1997}{michalewicz_1994}.

\section{Programação Evolucionária}

Desenvolvida por Lawrance Fogel na década de 1960, a programação evolucionária (PE) foi construída sobre diversos experimentos visando a previsão, sob algum critério arbitrário, de séries temporais não estacionárias através da evolução simulada dos estados das máquinas dentro de um limite de estados predeterminados, ou seja, dado os estados passados, previa-se os estados da máquina resultantes deste processo. Fogel buscou seguir um caminho diferente de pesquisa em relação ao que os trabalhos em inteligência artificial se concentravam à época, uma simulação primitiva de redes neurais. Para Fogel, havia uma grande limitação dos modelos baseados na inteligência humana em relação aos processos de criaturas com desenvolvimento contínuo do intelecto, necessário para sobrevivência em um dado ambiente (evolução).

Segundo \citet[pg.A2.3:3]{back_handbook_1997}, Fogel apresenta as primeiras tentativas de \enquote{[...](i) usar a evolução simulada para realizar predições, (ii) incluir codificações de comprimento variável, (iii) usar representações que tomam a forma de uma sequência de instrução, (iv) incorporar uma população de soluções candidatas e (v) coevoluir programas evolutivos} e partindo da premissa que \enquote{a inteligência é baseada na adaptação do comportamento para atingir metas em uma variedade de ambientes} (tradução nossa)\footnote{\textit{[...] (i) use simulated evolution to perform prediction, (ii) include variable-length encodings, (iii) use representations that take the form of a sequence of instructions, (iv) incorporate a population of candidate solutions, and (v) coevolve evolutionary programs [...] considered intelligence to be based on adapting behavior to meet goals in a range of environments.}}.

Devido ao contexto computacional, esses ambientes eram representados através de uma sequência de símbolos ou caracteres de um alfabeto finito arbitrário e o problema evolutivo definido como uma sequência de instruções, ou algoritmo, aplicadas sobre o conjunto de símbolos já observados. Com isso, ao inserir um conjunto de máquinas (população) em um dado ambiente, onde cada máquina possui um valor definido como o valor de entrada, esperava-se melhorar a performance de previsão do algoritmo, à medida que o valor de saída, ou o resultado, era comparado com o próximo valor de entrada. A qualidade desta previsão era, então, medida por uma função de recompensa que indicava o quanto cada máquina da população se adaptou ao ambiente.

Cada máquina pai pode criar um ou mais descendentes, onde cada descendente é criado pelo processo aleatório de alteração de estado, ou valor, do pai. Esse processo de mutação pode ocorrer sob uma certa distribuição de probabilidade ou ser definido no início da implementação do algoritmo. Ao fim de cada geração, a função de recompensa é aplicada sobre o descendente, assim como foi feito com seu pai, para avaliar o quão apto está em relação ao ambiente. As máquinas que fornecem o maior valor de recompensa permanecem no ambiente e se tornam pais das máquinas da geração seguinte. Este processo acontece sucessivas vezes até o símbolo que o símbolo que se deseja prever seja, efetivamente, previsto. A melhor máquina irá gerar essa previsão e esse novo símbolo é adicionado na população para ser avaliado no ambiente, reiniciando, assim, o processo.

Esse algoritmo foi aplicado com êxito em problemas de previsão, identificação e controle automático, simulação de coevolução de populações, experimentos de previsão de sequência, reconhecimento de padrões e em jogos. Na década de 1980, o algoritmo estendeu-se para novas aplicações, como no ordenamento de itens no problema do caixeiro viajante e em funções de otimização contínua, evoluindo, posteriormente, para implementações em planejamento de rotas, seleção ótima de subconjuntos e treinamento de redes neurais. No início da década de 1990, ocorre a primeira conferência anual de programação evolucionária, com exemplos de diversas aplicações de otimização na área de robótica, planejamento de caminhos e rotas, desenho e treinamento de redes neurais, controle automática entre outros \citeapud{back_handbook_1997}{michalewicz_1994}.

\section{Estratégias Evolutivas} 

Na metade da década de 1960, Bienert, Rechenberg e Schwefel, três estudantes da Universidade Técnica de Berlim, estudavam modelos de otimização aplicados em problemas da área de aerotecnologia e tecnologia espacial. Uma das principais pesquisas que realizavam à época, era de um robô experimental que, em um túnel de vento, deveria realizar uma série de testes em uma estrutura tridimensional fina e flexível visando minimizar a resistência em relação ao ar. Os primeiros testes não obtiveram sucesso. Foi apenas no ano seguinte ao início dos testes que \cite{rechenberg_cybernetic_1965} decide utilizar um dado para decisões aleatórias elaborando, assim, a primeira versão de uma EE \cite[pg.A2.3:6]{back_handbook_1997} chamada, posteriormente, de $(1 + l) EE$. 

Essa primeira versão consiste em uma sequência de instruções projetadas para otimização contínua bastante similar à busca aleatória, exceto por uma regra para a força da mutação conhecida como \enquote{regra do sucesso $\frac{1}{5}$} para ajuste do desvio padrão dessa mutação. Como a notação sugere, a estratégia de evolução $(1 + l)$ possui apenas um indivíduo pai que irá gerar apenas um indivíduo filho, onde ambos são confrontados e o indivíduo que representa a solução mais fraca, morre. Este indivíduo sobrevivente gera um novo filho e, assim, repetindo essa sequência diversas até se chegar a uma solução ótima. Sendo executado indivíduo a indivíduo, esse processo é computacionalmente custoso, apresentando uma convergência lenta para uma solução ótima, assim como tem a possibilidade de convergir para uma solução local.

Devido aos problemas de desempenho, os autores trabalharam em melhorias na estrutura do algoritmo, desenvolvendo uma nova versão chamada de EE multi-membro\footnote{\textit{EE multimembered}.} com população maior que 1. Novas melhorias foram realizadas nessa nova versão, resultando em dois princípios principais: $(\mu+1)$ e $(\mu, 1)$. No primeiro, $\mu$ indivíduos produzem $\lambda$ descendentes, gerando uma população temporária de $(\mu + \lambda)$ novos indivíduos, havendo a seleção de $\mu$ indivíduos para a geração seguinte. No segundo tipo, $\mu$ indivíduos produzem $\lambda$ descendentes, com $\mu < \lambda$, onde a nova população de $\mu$ indivíduos possui apenas os indivíduos selecionados do conjunto de $\lambda$ descendentes, limitando o tempo de vida de um indivíduo apenas a uma geração específica.

A partir da primeira versão, a comunidade de pesquisadores da área de EE realizaram novas aplicações nas décadas seguintes que não se reduziram somente ao objetivo da otimização de valores do mundo real, como a aplicação para otimização binária em estruturas de indivíduos multicelulares usando a ideia de sub populações, estratégias evolutivas para problemas com multi critérios, entre diversas outras aplicações seguindo a ideia principal de melhoria contínua dos indivíduos analisados \Citep{back_handbook_1997}.

\section{Algoritmos Genéticos} 
\label{sec:AlgoritmosGeneticos}

No início da década de 1960, na Universidade de Michigan, John H. Holland\footnote{John Henry Holland foi um bacharel em Matemática pelo Instituto de Tecnologia de Massachusetts, com mestrado na mesma área. Em 1959, é a primeira pessoa a receber o título de Ph.D (Doutor) da Universidade de Michigan em Ciência da Computação. Holland foi um dos responsáveis pela criação do primeiro Centro de Estudos de Sistemas Complexos, em Michigan. Na segunda metade da década de 1980, se torna uns dos principais membros do Instituto Santa Fe, no Novo México, que pesquisava fenômenos não-lineares. Da conquista de seu título de Doutor até seu falecimento, em 9 de agosto de 2015, permaneceu como professor na Universidade de Michigan lecionando nas disciplinas de Economia, Biologia e Ciência da Computação \citep{britannica_2022}.} e seus alunos davam os primeiros passos nos estudos de operadores genéticos para resolução de problemas em ambientes artificiais. Previamente aos primeiros estudos de Holland, é válido citar que o uso de computadores para simular sistemas artificiais ou genéticos com o objetivo de observar fenômenos naturais já era tema de pesquisa em trabalhos de biólogos que buscavam entender como a interação entre os indivíduos de uma população impactavam nas gerações subsequentes \citep[p.90]{goldberg_genetic_1989}.

Em 1962, em seu artigo intitulado \enquote{\textit{Outline for a Logical Theory of Adaptive Systems}}, Holland apresenta as primeiras ideias de sua Teoria de Sistemas Adaptativos que, conforme o autor apresenta, tinha o seguinte objetivo \citep{holland_1962}:

\citacao
	{%
		[...] delinear uma teoria de autômatos adequada às propriedades, requisitos e questões de adaptação. As condições que tal teoria deve satisfazer vêm não de um, mas de vários campos: deve ser possível formular, pelo menos em uma versão abstrata, algumas das principais hipóteses e problemas de partes relevantes da biologia, particularmente as áreas relacionadas ao controle molecular e neurofisiologia. [...] Em linhas gerais, é um estudo de como os sistemas podem gerar procedimentos que lhes permitam se ajustar de forma eficiente aos seus ambientes. Para que a adaptabilidade não seja arbitrariamente restringida no início, o sistema de adaptação deve ser capaz de gerar qualquer método ou procedimento capaz de uma definição efetiva. [...] O processo de adaptação pode, então, ser visto como uma modificação do processo de geração à medida que as informações sobre o ambiente se acumulam. Isso sugere que os sistemas adaptativos sejam estudados em termos de classes associadas de procedimentos de geração – a classe associada em cada caso sendo o repertório do sistema adaptativo. [...] A adaptação, então, baseia-se na seleção diferencial de programas de supervisão. Ou seja, quanto mais "bem-sucedido" for um programa de supervisão, em termos da capacidade de seus programas de resolução de problemas produzirem soluções, mais predominante ele se tornará (em números) em uma população de programas de supervisão.
	}{%
		[...] to outline a theory of automata appropriate to the properties, requirements and questions of adaptation. The conditions that such a theory should satisfy come from not one but several fields: It should be possible to formulate, at least in an abstract version, some of the key hypotheses and problems from relevant parts of biology, particularly the areas concerned with molecular control and neurophysiology. [...] In general terms, it is a study of how systems can generate procedures enabling them to adjust efficiently to their environments. If adaptability is not to be arbitrarily restricted at the outset, the adapting system must be able to generate any method or procedure capable of an effective definition. [...] The process of adaptation can then be viewed as a modification of the generation process as information about the environment accumulates. This suggests that adaptive systems be studied in terms of associated classes of generation procedures--the associated class in each case being the repertory of the adaptive system. [...] Adaptation, then, is based upon differential selection of supervisory programs. That is, the more "successful" a supervisory program, in terms of the ability of its problem-solving programs to produce solutions, the more predominant it is to become (in numbers) in a population of supervisory programs.
	}
	{\citep[p.297-298]{holland_1962}}
	{(tradução nossa).}

Dessa forma, Holland buscou nas décadas seguintes desenvolver uma teoria com os procedimentos necessários para criação de máquinas e programas computacionais que pudessem resolver problemas gerais com alta capacidade de adaptação a ambientes complexos através de alguns operadores que selecionariam, dentre um conjunto de programas ou máquinas, o mais apto à sobrevivência. Em outras palavras, os mais bem-sucedidos em encontrar uma solução para um dado problema, tinha maior chance de ser escolhido e novamente testado.

Segundo \Citet[pg.A2.3:4]{back_handbook_1997}, Holland \enquote{[...] estabeleceu uma agenda ampla e ambiciosa para compreender os princípios subjacentes dos sistemas adaptativos – sistemas que são capazes de automodificação em resposta às suas interações com os ambientes em que devem funcionar} (tradução nossa)\footnote{\textit{Holland set out a broad and ambitious agenda for understanding the underlying principles of adaptive systems—systems that are capable of self-modification in response to their interactions with the environments in which they must function.}}. Para \Citet[p. 1]{goldberg_genetic_1989}, Holland tinha dois objetivos principais em sua pesquisa: uma explicação bem estruturada e fundamentada dos processos de adaptação de sistemas naturais e a construção de programas computacionais de sistemas artificiais com a finalidade de incorporar importantes mecanismos destes sistemas, sendo o foco da pesquisa a robustez dos algoritmos, ou seja, o equilíbrio entre a eficiência e a eficácia necessária para a sobrevivência de possíveis soluções em muitos ambientes diferentes.

Diferentemente dos estudos apresentados anteriormente de algoritmos aplicados em modelos de previsão ou otimização, Holland se debruçou sobre modelos evolutivos para entendimento de sistemas adaptativos robustos naturais e projeção de elementos adaptativos em um dado contexto. Para o autor, em sistemas adaptativos naturais, as características relativas à competição e inovação entre os agentes ao longo destes processos naturais eram fundamentais para que os indivíduos se adaptassem ao ambiente e às mudanças imprevistas que este aplicava sobre os indivíduos \Citep{back_handbook_1997}.

O grande deferencial da linha desenvolvida por Holland, foi a incorporação de diversos conceitos da genética (fenótipo, genótipo, reprodução, cruzamento, mutação, entre outros) que se demonstraram altamente eficientes e performáticos na resolução de problemas complexos utilizando poucos dados de entrada, assim como os processos de busca para encontrar soluções ótimas apresentavam inovações em relação à resolução de problemas e aprendizados dos elementos no ambiente ao longo dos processos evolutivos. Dessa forma, como veremos a seguir, a partir de suas primeiras publicações, a Teoria de Sistemas Adaptativos foi se tornando o objeto de estudo de cada vez mais pesquisadores, conquistando um importante espaço no campo de pesquisa da Inteligência Artificial e absorvendo novas ideias das diversas aplicações realizadas, em um grande número de áreas, nas décadas seguintes.

\subsection{A Evolução e as Contribuições Gerais para a Área}
\label{subsec:EvolucaoCampoPesquisaAlgoritmosGeneticos}

\citet{bagley_1967} publica a primeira aplicação, externa ao grupo de pesquisa liderado por Holland, utilizando os conceitos da Teoria de Sistemas Adaptativos. É em seu trabalho que aparece pela primeira vez o termo \enquote{algoritmo genético} \cite[pg.133]{bagley_1967}. Em sua tese, Bagley elabora um programa de testes controlável para aplicação de tarefas a um jogo de \textit{hexapawn}\footnote{Hexapeão, em tradução livre, é um jogo para dois jogadores jogado em um tabuleiro 3 x 3, onde cada jogador começa com 3 peões e tem o objetivo de chegar até a outra extremidade ou impedir o avanço de seu oponente.}. Com um AG que busca por um conjunto de parâmetros através de funções de avaliação do jogo realizado e os compara com outros algoritmos de correlação, o programa ajusta os novos procedimentos conforme os resultados de cada jogo. Segundo \Citet[pg.93]{goldberg_genetic_1989}, Bagley introduz dois conceitos importantes à Teoria: um mecanismo de ajuste da adaptação que reduz as chances de uma solução ser escolhida muito cedo, havendo uma \enquote{super solução} que domina as demais, e que aumenta as chances de seleção de soluções com maior valor de aptidão nas gerações seguintes, aumentando a competitividade; a noção de um AG que se autorregula, sugerindo a codificação das probabilidades de cruzamento e mutação dentro do cromossomo (solução).

No mesmo período, com ênfase nos aspectos biológicos, \Citet{rosenberg_1967}, utilizando um AG, simula uma população de organismos monocelulares em busca dos organismos com as propriedades bioquímicas mais ajustadas a uma estrutura genética arbitrária. Em sua tese, com o cálculo de funções não lineares, Rosenberg faz a primeira aplicação de um AG multiobjetivo que busca o valor máximo de adaptação de uma estrutura específica através da minimização do impacto de organismos com baixo valor de adaptação. 

Absorvendo os novos métodos implementados por Bagley e Rosenberg, \Citet{cavicchio_adaptive_1970} implementa AGs em problemas de seleção de subrotina e problemas de reconhecimento de padrões, inovando com a inserção de um método que ele chamou de esquema de pré seleção, onde uma prole com um bom valor de aptidão substitui seu pai com o objetivo de manter uma população mais diversificada. Com isso, Cavicchio apresenta um dos primeiros estudos sobre formas elitistas de seleção e adaptação de taxas de cruzamento e mutação. No mesmo ano, \Citet{weinberg1969computer}, assim como Rosenberg, aplica AGs para a busca dos cromossomos que melhor se adaptam à uma estrutura genética. Contudo, diferentemente dos trabalhos citados anteriormente, Weinberg sugere a aplicação de um AG para a ajuste dos parâmetros de um AG de nível inferior, chamando o AG de nível mais alto de um programa genético não adaptativo e o AG de nível inferior, onde os parâmetros são ajustados, de programa genético adaptativo. Em outras palavras, o AG de nível superior determina a direção em que o AG de nível inferior evoluirá. Em 1967, Holland começa a elaborar  as primeiras ideias de esquemas de sistemas adaptativos e, em 1969, demonstra aplicações de alocação ótima utilizando o modelo de k-bandidos armados \citeapud{holland_1967}{back_handbook_1997}.

É apenas em 1971 que AGs são aplicados a problemas reais. \Citet{hollstien1971artificial} implementa AGs há um conjunto de 14 problemas de otimização matemática em busca de qual ou quais soluções melhor otimizam o controle de retorno digital de uma determinada planta de engenharia. Uma de suas principais contribuições, foi a observação do impacto negativo que uma população pequena (no caso de sua tese, uma população de 16 indivíduos) tem na robustez do algoritmo e, com isso, recomendando a utilização de populações maiores para futuros testes de otimização. \Citet{frantz_1972} leva em consideração a recomendação de Hollstien. Em seu trabalho, Frantz aplica AGs em uma população de tamanho $n = 100$ vetores de comprimento $l = 25$, com o objetivo de estudar os impactos que a ordem dos genes em um cromossomo, assim como a mudança de posição dos blocos de construção internos do cromossomo, tinha na otimização de AG. As funções configuradas por Frantz para avaliar tais impactos foram não foram robustas o bastante na captação dos impactos, não demonstrando, assim, maiores variações de performance de um ordenamento em relação à outro, considerando o experimento inconclusivo. Contudo, como apontado por \Citet[pg.102]{goldberg_genetic_1989}, Frantz faz duas importantes contribuições através da introdução do operador de complemento parcial e do operador de cruzamento de múltiplos pontos.

Em 1975, Holland reuni as ideias desenvolvidas em sua pesquisa até aquele momento e publica seu principal trabalho, o livro \enquote{\textit{Adaptation in Natural and Artificial Systems}} \Citep{holland_adaptation_1975}. No mesmo ano, \Citet{de_jong_1975}\footnote{Kenneth Allan De Jong foi um dos alunos mais célebres de Holland. Sob sua orientação, recebe seu título de doutor em Ciência da Computação pela Universidade de Michigan em 1975. Com suas pesquisas, De Jong contribuiu para áreas dos AGs, EC, \textit{machine learning} e sistemas adaptativos complexos. Atualmente, é professor emérito em Ciência da Computação na Universidade de Goerge Mason, Virgínia, e um dos principais pesquisadores em CE.} publica sua tese intitulada \enquote{\textit{Analysis of the behavior of a class of genetic adaptive systems}}, uma das publicações mais importantes no campo de estudos dos AGs. Segundo \Citet[pg.107]{goldberg_genetic_1989}, \enquote{O estudo de De Jong permanece um marco no desenvolvimento de algoritmos genéticos devido à combinação da Teoria de Esquema de Holland e seus próprios experimentos computacionais cuidadosos}\footnote{\textit{De Jong's study still stands as a milestone in the development of genetic algorithms because of its combination of Holland's theory of schemata and his own careful compuational experiments.}} (tradução nossa).

De Jong via os AGs como uma ferramenta poderosa e bastante flexível na resolução de um grande número de problemas complexos. Em sua tese, contudo, foca na aplicação de AGs para a otimização de funções, onde descreve minuciosamente os processos realizados pelos AGs na resolução de problemas de otimização, além de refinar e simplificar a configuração de outros elementos importantes como o ambiente e os critérios de desempenho. 

Em um ambiente de testes, De Jong implementou AGs em 5 problemas de minimização de função, avaliando a efetividade de cada AG através de duas métricas: um medidor de convergência e um medidor de convergência contínuo. A primeira foi chamada de medida de performance \textit{off-line} (convergência) enquanto a segunda de medida de performance \textit{on-line} (contínua). O uso de cada métrica dependerá do contexto de aplicação, onde, em uma aplicação \textit{off-line}, era realizada a simulação de avaliação de funções, onde a função com o melhor resultado (média dos melhores valores de performance em cada momento do tempo) seria a escolhida após alcançar um critério de parada predeterminado. Em relação a uma aplicação \textit{on-line}, diferentemente da anterior, a avaliação das funções não é realizada através de simulações, mas sobre dados reais, obtendo, como resultado (média de todas as funções avaliadas), uma recompensa conforme o desempenho da função avaliada pela métrica.

Após definido o ambiente de testes e as métricas de desempenho, De Jong implementou uma primeira versão de um AG que chamou de $R1$ (\textit{reproductive plan 1}), similar ao AG simples apresentado na \autoref{sec:componentes_de_um_algoritmo_genetico}, onde, seguindo um processo aleatório, 3 operadores principais são aplicados sobre uma população de cadeias codificadas de caracteres, sendo eles: i) Operador de reprodução ou seleção; ii) Operador de cruzamento; iii) Operador de mutação. Nesta primeira versão, eram inseridos quatro parâmetros de entrada como o tamanho da população ($n$), a probabilidade de cruzamento ($p_c$), a probabilidade de mutação ($p_m$) e a lacuna de geração ($G$)\footnote{\textit{Generation gap}, ou lacuna de geração em tradução livre, não será um operador que iremos nos aprofundar nestes trabalho. Contudo, em termos gerais, é um método que permite que uma população não sobreponha a outra, convergindo para um resultado muito rápido.}. Os resultados deste primeiro experimento demonstraram algumas características importantes do AG aplicado. Populações maiores tendem a apresentar uma performance melhor em ambientes simulados, em relação à dados reais, devido à alta diversidade entre os indivíduos. No entanto, populações pequenas têm capacidade de adaptação mais rapidamente, apresentando um bom desempenho nas gerações iniciais. O operador de mutação é um importante elemento que diminui a perda de alelos ao longo das gerações, mantendo uma diversidade populacional suficiente para melhoria contínua das populações. Em relação ao operador de reprodução, o autor sugeriu a aplicação de uma probabilidade $p_c = 0,6$ como um valor que equilibrasse a performance tanto no contexto de uma aplicação \textit{off-line} quanto \textit{on-line}. Em outras palavras, para cada 10 indivíduos em uma população, 6 seriam selecionados para cruzamento.

No plano $R2$ (um modelo elitista), De Jong identificou que em superfícies unimodais, o modelo apresentou um crescimento significativo na performance em aplicações \textit{on-line} e \textit{off-line}. Contudo, cruzando estes resultados com o experimento do modelo $R5$ (um modelo de fator de aglomeração), foi identificado que o elitismo nos AGs faz com que o modelo melhore sua performance em relação à busca local em detrimento à busca por ótimos globais. No modelo $R3$ (um modelo de valor esperado), De Jong faz uma outra importante contribuição. Com o objetivo de reduzir os erros estocásticos da seleção aleatória, o autor inseriu no modelo o cálculo do número esperado de proles de cada geração, indicando o quão distante o número de indivíduos simulados ficou do resultado esperado.

Com isso, a segunda metade da década de 1970 foi um marco no campo de pesquisa dos AGs. O interesse pela área foi crescendo progressivamente nos anos seguintes. Em 1976, pesquisadores da Universidade de Michigan, da Universidade de Pittsburgh, entre outras, organizaram a primeira conferência de sistemas adaptativos. Em 1979, Holland, De Jong e Sampson escalam o tamanho da conferência através de um financiamento para realizarem uma conferência interdisciplinar em sistemas adaptativos, que acabou sendo realizado em 1981 na Universidade de Michigan. Em 1985, na Universidade de Pittsburgh, ocorre a primeira Conferência Internacional sobre Algoritmos Genéticos (ICGA) que, devido ao sucesso, passou a ser semestral nos próximos anos. Em 1989, surge a Sociedade Internacional de Algoritmos Genéticos (ISGA), organização responsável pelo financiamento de conferências e atividades das áreas de pesquisas relacionadas aos AGs, tendo como uma de suas primeiras conquistas a criação de uma das principais conferências da comunidade, sobre os Fundamentos dos Algoritmos Genéticos (FOGA) \cite[pg.A2.3:5]{back_handbook_1997}.
	% Computação Evolucionária
\chapter{Algoritmos Genéticos}
\label{chap:altoritmos_geneticos}

Neste capítulo, procura-se introduzir, aprofundar e discutir os principais componentes, e suas características, para a construção de um algoritmo genético simples.

\section{Introdução aos Algoritmos Genéticos}
\label{sec:introducao_aos_algoritmos_geneticos}

Um algoritmo genético é uma meta-heurística\footnote{Heurística é uma técnica construída para encontrar, gerar ou selecionar uma solução para um problema específico dentro de um problema maior definido, sendo a meta-heurística, então, uma heurística de alto nível que busca por heurísticas para resolução de um dado problema principal.} com finalidades variadas que, conforme citado por \citet[pg.27]{MelanieMitchell98}, pode ser dividido em dois campos principais de aplicação: como uma técnica de busca de possíveis soluções de problemas tecnológicos e como um modelo computacional que objetiva simular sistemas naturais em busca de respostas como, por exemplo, um maior entendimento dos processos evolutivos e de seleção natural. No primeiro, a gama de aplicações é extensa, encontrando-se diversos trabalhos em problemas das ciências exatas e ciências sociais; em relação ao segundo, encontram-se diferentes empregos de algoritmos genéticos nas áreas das ciências biológicas. Neste trabalho, serão abordadas as aplicações relativas a problemas nas ciências sociais, em especial, nas ciências econômicas.

Relativo às ciências econômicas, dentre as diversas aplicações encontradas na literatura, buscar-se-á abordar três principais: a busca por soluções ótimas de problemas de otimização, a procura por padrões ou características que podem ser entendidas como inovações em um dado processo ou contexto e, por último, o aprendizado que pode ser extraído dos processos de um AG através da aplicação de seus operadores e interação entre os elementos analisados. No presente capítulo, com o objetivo de abordar cada operador e processo de um AG simples, utilizar-se-á um exemplo de aplicação que busca encontrar uma solução de um problema de otimização. As demais aplicações apresentadas acima serão aprofundadas no capítulo seguinte.

De forma geral, um AG funciona da seguinte forma: definido um problema, o algoritmo realiza uma busca por uma solução global em um espaço de possíveis soluções, onde, ao realizar essa busca, ele pode encontrar ou não uma solução ótima. Inicialmente, estas possíveis soluções são construídas de forma aleatória e combinam suas características entre si, formando novas possíveis soluções. Tais características determinarão como e quais serão os atributos combinados ou ignorados neste processo, que é realizado de forma iterativa\footnote{Na programação, é uma ação que se repete sucessivamente até atingir um resultado desejado ou alguma ordem de término.}, onde cada iteração termina com novas soluções construídas através da troca de informações entre os elementos. Estas sucessivas iterações terminam quando o objetivo predefinido é alcançado ou o algoritmo não encontra nenhuma solução que satisfaz o problema. Um AG básico, ou seja, que contenha pelo menos a utilização dos operadores de reprodução, cruzamento e mutação, é de simples construção e parametrização. Embora simples, é capaz de procurar por soluções em um espaço de busca muito maior e um desempenho acima dos programas convencionais \citet[pg.66]{holland_genetic_1992} e, conforme cita \citet[pg.2]{goldberg_genetic_1989}, podem ser divididos em três métodos de busca principais: baseado em cálculo, enumerativo e aleatório ou randômico.

O primeiro tipo, é uma heurística de busca local e é subdividido em duas classes: indireto e direto. Através da resolução de, normalmente, um conjunto de equações não lineares, as técnicas de busca indireta procuram por extremos locais resultantes de uma função objetivo igual a zero. Em outras palavras, conforme a direção definida pelo vetor gradiente, analisa-se se o ponto de aclive ou declive, que não possui mais nenhuma variação para qualquer direção, assume o valor de máximo ou mínimo com base nas funções calculadas. Em relação aos métodos diretos, com uma solução arbitrária inicial, são feitos repetidos incrementos nesta solução e, caso essa mudança apresente um resultado melhor, é feito um novo incremento, e assim sucessivamente, até não haver mais nenhuma melhoria, levando-se em consideração as restrições do problema.

O segundo método, uma heurística de busca global, inicia buscando pelos valores da função objetivo em cada um dos pontos dentro de um espaço de busca delimitado ou um espaço de busca infinito discreto, parando ao encontrar um extremo global que se apresente como uma possível solução para o problema a ser resolvido. 

Por último, o método de busca aleatória, também uma heurística de busca global, utiliza algum tipo de aleatoriedade ou probabilidade na busca por ótimos globais dentro de um espaço definido.

\citet[pg.5]{goldberg_genetic_1989} também cita que os métodos acima apresentados, com um exemplo de extremo fictício ilustrado na \autoref{fig:SinglePeakCalculusBasedMethod}, foram perdendo a relevância ao longo do tempo, pois são métodos de busca úteis em um número muito pequeno de situações, não apresentando uma eficácia e eficiência satisfatórias na resolução de um espectro grande de problemas, de diferentes níveis de complexidade, conforme a realidade demanda. Dessa forma, os algoritmos genéticos foram se destacando por sua robustez na resolução de um número considerável de problemas de otimização através da adaptação para os sistemas artificiais de alguns conceitos da biologia e da genética, sobretudo, o conceito darwiniano de evolução dos indivíduos mais aptos.

\figura
	{Exemplo de um extremo local}
	{.5}
	{imagens/SinglePeakCalculusBasedMethod.png}
	{SinglePeakCalculusBasedMethod}
	{
		\citet[p.3]{goldberg_genetic_1989}
	}

Dessa forma, nas seções subsequentes, serão explorados os principais conceitos, operadores e processos para a construção de um algoritmo genético simples.

\section{Algoritmos e Linguagem de Máquina}
\label{sec:algoritmos_e_linguagem_de_maquina}

Para \citet[pg.2]{cormen_introduction_2009}, um algoritmo é um \enquote{procedimento computacional bem definido que recebe um valor, ou um conjunto de valores, como entrada e produz algum valor, ou conjunto de valores, como saída}\footnote{[...] a well-defined computational procedure that takes some values, or set of values, as input and produces some value, or set of values, as output.} (tradução nossa). Não é diferente com um AG, que, dada uma função de otimização, depende de um conjunto de parâmetros de entrada para encontrar um, ou mais de um, ponto ótimo, ou próximo ao ótimo, como valor, ou valores, de saída.

Sendo o desenho do AG uma simulação de um processo natural, é necessário que haja uma codificação dos valores de entrada para que o sistema computacional possa processá-los. Ou seja, é necessário realizar uma transformação das informações que os humanos interpretam, modificam e constroem, com base nos estudos dos processos naturais, em uma linguagem que o computador entenda, chamada linguagem ou código de máquina. 

Como apresentado por \citet[pg.42]{fedeli_introducao_2009}, atualmente, os computadores utilizam apenas dois operadores básicos em sua linguagem, sendo eles os dígitos binários 0 e 1, também conhecidos como bit (do inglês, \textit{binary digit}). O bit é a menor informação armazenada pela memória e processada pela unidade central de processamento (UCP) de um computador, onde, em um determinado espaço da memória, é armazenado um e somente um bit (0 ou 1) por vez. De forma mais ilustrativa, pode-se entender o 0 como uma instrução para um corte de energia ou uma informação relativa à negação, impedimento ou inexistência; do contrário, o 1 é uma instrução para passagem de energia ou uma informação relativa à positivação, desimpedimento ou existência

Existem várias formas de agrupamento destes dígitos, havendo interpretações e funções diferentes levando-se em consideração a quantidade e a ordem dos bits nestes agrupamentos, sendo o \textit{American Standard Code for Information Interchange (ASCII)} o método de armazenamento e representação de caracteres mais utilizado pelas plataformas de computadores pessoais \citep[pg.46]{fedeli_introducao_2009}, onde cada dígito é formado pela junção de 8 bits\footnote{Originalmente, o conjunto mínimo era de 7 bits \citet{gorn_american_1963}.}, unidade conhecida como byte (do inglês, \textit{binary term}). Dessa forma, quando o dígito \enquote{A}, por exemplo, é enviado para o computador, o ASCII o codifica, enviando a sequência $01000001$ para armazenamento na memória. O mesmo processo acontece de forma inversa, quando o computador gera algum dígito que precisa ser representado graficamente.

Com isso, por questões relativas à facilidade de interpretação e desempenho computacional, os parâmetros de entrada de um AG são codificados em sequências de 0s e 1s, onde cada sequência tem origem de um determinado alfabeto \footnote{Aqui, alfabeto nada mais é que um conjunto finito de algorismos ou caracteres.}, que permite realizar a codificação e decodificação dos valores de entrada e saída, respectivamente, do algoritmo aplicado.

\section{Componentes de um Algoritmo Genético}
\label{sec:componentes_de_um_algoritmo_genetico}

A forma como os parâmetros de um AG são definidos impacta diretamente na robustez de seu funcionamento e das possíveis soluções encontradas. Dessa forma, serão explorados a seguir os principais elementos, e suas características, para a construção de um AG simples. Para isso, será utilizado como exemplo o Problema da Caixa Preta\footnote{\label{rodape:problema_caixa_preta}Através de uma máquina ou dispositivo que possui um número específico de parâmetros de entrada (interruptores), o Problema da Caixa Preta consiste em obter o maior valor de saída possível com base na configuração destes parâmetros. O objetivo é analisar como um determinado sistema fechado relaciona os valores de entrada e a resposta, com base no estímulo destes valores, de saída.} apresentado por \citet[p.8]{goldberg_genetic_1989}.

\subsection{Alelo e Locus}
\label{subsec:alelo_e_locus}

Conforme abordado anteriormente (ver \autoref{sec:algoritmos_e_linguagem_de_maquina}), o primeiro passo na construção de um AG é a codificação dos valores de entrada em um conjunto de 0s e 1s. Cada valor, é uma característica binária ou um detector chamado de alelo (equivalente ao gene de um cromossomo na biologia). A posição de cada caractere dentro deste conjunto é chamada de \textit{locus}. Pode-se analisar o locus à parte do gene (ou alelo). Por exemplo, supondo que as características (genes) do cabelo de um ser humano podem ser encontradas no cromossomo 2. Ao analisar os genes relativos ao cabelo dentro deste cromossomo, identifica-se no locus 3 (posição 3) que a cor do cabelo é preta (valor do alelo). Com isso, temos que o valor e a posição de cada elemento nesse conjunto apresentará uma característica específica, seja olhando para um único valor ou para um subconjunto de valores.

\subsection{Cadeia de Caracteres}

O conjunto codificado dos valores de entrada de um AG é chamado de cadeia de caracteres\footnote{Do inglês, \textit{string}.} ou vetor. Comparativamente, na biologia, o cromossomo é uma estrutura constituída por DNA (ácido desoxirribonucleico), sendo cada DNA composto por um número de genes que, por sua vez, são responsáveis por definirem a(s) característica(s) de um indivíduo. Dessa forma, temos que o vetor de um AG é um elemento artificial análogo ao cromossomo nos sistemas naturais. Como exemplo, o vetor com os possíveis valores da Caixa Preta pode ser representado da seguinte forma:

\tabelasimples
	{%
		| >{\centering\arraybackslash}m{2.5cm}   % Coluna 1
		| >{\centering\arraybackslash}m{2cm}   	 % Coluna 2
		| >{\centering\arraybackslash}m{2cm}     % Coluna 3
		| >{\centering\arraybackslash}m{2cm}     % Coluna 4
		| >{\centering\arraybackslash}m{2cm}     % Coluna 5
		| >{\centering\arraybackslash}m{2cm} |   % Coluna 6
	}
	{Cadeia de Caracteres ou Vetor Contendo 5 Genes}
	{%
		Característica Binária (Gene) & Interruptor 1 & Interruptor 2 & Interruptor 3 & Interruptor 4 & Interruptor 5 \\ \hline % Header
		Valor da Característica Binária (Alelo) & $0$ & $1$ & $1$ & $0$ & $1$ \\ \hline % Linha 1
		Posição ou Índice do Gene (Locus) & $1$ & $2$ & $3$ & $4$ & $5$ \\ \hline % Linha 2
		Referência Caixa Preta & Desligado & Ligado & Ligado & Desligado & Ligado \\ \hline
	}
	{MatrizCadeiaCaracteresAleloLocus}
	{adaptado de \citet[p.11]{goldberg_genetic_1989}}

\subsection{População Inicial e Gerações}

Inicialmente, o AG começa a realizar sua busca sobre um conjunto aleatório de indivíduos (cadeias de caracteres ou vetores), chamado de população inicial. Através dessa busca, o algoritmo analisa cada um dos elementos da população, identificando quais são os mais aptos a sobreviver levando-se em consideração os demais indivíduos e o ambiente em que estão localizados. Esta estrutura composta por vários indivíduos com genes específicos se manterá por toda a vida do organismo (na genética, esta estrutura é chamada de genótipo) e, à medida que as gerações (iterações) passam, os indivíduos interagem entre si e com o ambiente criando, assim, um novo organismo. Na genética, a nova estrutura resultante deste processo é conhecida como fenótipo.

\tabelasimples
	{%
		| >{\centering\arraybackslash}m{2.5cm}   % Coluna 1
		| >{\centering\arraybackslash}m{2cm}   	 % Coluna 2
		| >{\centering\arraybackslash}m{2cm}     % Coluna 3
		| >{\centering\arraybackslash}m{2cm}     % Coluna 4
		| >{\centering\arraybackslash}m{2cm}     % Coluna 5
		| >{\centering\arraybackslash}m{2cm} |   % Coluna 6
	}
	{População de tamanho 4}
	{%
		Indivíduo (População) & Interruptor 1 & Interruptor 2 & Interruptor 3 & Interruptor 4 & Interruptor 5 \\ \hline % Header
		Vetor 1 & $0$ & $1$ & $1$ & $0$ & $0$ \\ \hline % Linha 1
		Vetor 2 & $1$ & $1$ & $0$ & $0$ & $0$ \\ \hline % Linha 2
		Vetor 3 & $0$ & $1$ & $0$ & $0$ & $0$ \\ \hline % Linha 3
		Vetor 4 & $1$ & $0$ & $0$ & $1$ & $1$ \\ \hline % Linha 4
	}
	{PopulacaoTamanho4}
	{Elaborado pelo autor.}
	
\subsection{Paisagem de Aptidão, Sobrevivência do Mais Apto e Função Objetivo}

O ambiente em que os indivíduos (vetores) estão localizados é um elemento importante que compõe os processos de um AG, sendo um dos responsáveis por direcionar a escolha de quais indivíduos irão passar para a próxima geração e quais irão morrer. Ou seja, o algoritmo irá analisar cada indivíduo num dado espaço procurando os mais aptas a sobreviverem ao ambiente em que estão localizados com base em suas características. Este espaço de busca é conhecido como paisagem ou horizonte de aptidão\footnote{do inglês, \textit{fitness landscape}}, definido por \citeauthor{langdon_foundations_2002} da seguinte maneira:

\citacao
	{%
		Na forma mais simples, uma paisagem de aptidão pode ser vista como um gráfico onde cada ponto na direção horizontal representa todos os genes em um indivíduo (genótipo) correspondente àquele ponto. A aptidão daquele indivíduo é plotada como a altura. Se os genótipos podem ser visualizados em duas dimensões, o gráfico pode ser visto como um mapa de três dimensões, que pode conter montes e vales. Grandes regiões de baixa aptidão podem ser consideradas pântanos, enquanto grandes regiões de alta aptidão, que se mantêm num mesmo nível, podem ser consideradas como platôs.
	}{%
	In its simplest form a fitness landscape can be seen as a plot where each point in the horizontal direction represents all the genes in an individual (known as its genotype) corresponding to that point. The fitness of that individual is plotted as the height. If the genotypes can be visualised in two dimensions, the plot can be seen as a three-dimensional map, which may contain hills and valleys. Large regions of low fitness can be considered as swamps, while large regions of similar high fitness may be thought of as plateaus.
	}
	{\citep[pg.4]{langdon_foundations_2002}}
	{(tradução nossa).}

 Como ilustrado na \autoref{fig:2DFitnessLandscape}, um AG possui um espaço de busca predeterminado (área cinza) contendo a população inicial a ser analisada, representada na imagem pelo ponto preto mais inferior, à esquerda. Com base nas características binárias (genes) da população ou estrutura (genótipo), o AG determinará quais indivíduos passarão para a próxima geração (seta branca), formando uma nova população. Esse processo acontece sucessivamente até que, com base no estado (fenótipo) de cada nova população, seja encontrada a que possui as maiores chances de sobrevivência, sendo a possível solução para um dado problema de otimização.

\figura
	{Paisagem de Aptidão Representada em 2 dimensões}
	{.7}
	{imagens/2DFitnessLandscape.PNG}
	{2DFitnessLandscape}
	{\citet[p.5]{langdon_foundations_2002}}

Desse modo, como foi possível observar, o processo de busca pela população inicial, que é formada de forma randômica, e que determina como será formada a população na geração seguinte, possui grande peso da aleatoriedade, porém de forma diferente do tipo de busca aleatória ou randômica apresentada anteriormente (ver \autoref{sec:introducao_aos_algoritmos_geneticos}). Um dos grandes diferenciais do AG é a possibilidade de ter certo controle sobre seus processos aleatórios. Isso acontece, pois o algoritmo observa os dados históricos, que surgem a partir de cada nova geração, assim como segue alguns direcionamentos nas busca por novos pontos com maior chance de sobreviver ao ambiente. Estes direcionamentos são construídos por uma função objetivo (na biologia, chamada de função de aptidão\footnote{do inglês, \textit{fitness function}.}), sendo os seus parâmetros o que definirá quais serão os critérios de sobrevivência ou aptidão para que um ou mais indivíduos de uma população sigam para a geração seguinte, processo representado graficamente pelos pontos mais elevados na paisagem de aptidão na \autoref{fig:FitnessLandscapeAdaptado}.

\figura
	{Paisagem de Aptidão Representada em 3 dimensões}
	{.7}
	{imagens/FitnessLandscapeAdaptado.png}
	{FitnessLandscapeAdaptado}
	{adaptado de \citet[p.5]{langdon_foundations_2002}}

\subsection{Reprodução ou Seleção}
\label{subsec:reproducao_ou_selecao}

Dado um primeiro conjunto de indivíduos, chamado de população inicial, o AG precisa determinar quais destes indivíduos têm maior probabilidade de se adaptar ao ambiente e passar seus genes para a próxima geração através de seus descendentes. A esse processo é dado o nome de reprodução ou seleção.

A reprodução é um processo em que os indivíduos são replicados, copiados, conforme seus valores de aptidão, que são calculados pela função objetivo. Os indivíduos que possuem os maiores valores de aptidão têm maior probabilidade de criarem novos descendentes, passando, dessa forma, parte dos seus genes para a população seguinte. Nos sistemas naturais, o que determinará se um indivíduo passará por esse processo é sua capacidade de sobreviver a qualquer elemento que o possa matar antes que consiga se reproduzir, sendo ele um predador, uma doença, etc. (\citet[p.11]{goldberg_genetic_1989})

Conforme o exemplo da Caixa Preta apresentado na \autoref{sec:componentes_de_um_algoritmo_genetico}, foi sugerida uma população inicial, criada aleatoriamente, com os indivíduos (interruptores) 01101, 11000, 01000 e 10011 (ver \autoref{tab:MatrizCadeiaCaracteresAleloLocus}).

O primeiro passo será calcular o valor de aptidão de cada indivíduo, sua participação na população como um todo e ordená-los por valor de aptidão (\autoref{tab:ValoresAptidaoIndividuosPopulacaoIndividual}).

\tabelasimples
	{%
		| >{\centering\arraybackslash}m{1cm} % Coluna 1
		| >{\centering\arraybackslash}m{2cm} % Coluna 2
		| >{\centering\arraybackslash}m{2cm} % Coluna 3
		| >{\centering\arraybackslash}m{2.5cm} % Coluna 4
		| >{\centering\arraybackslash}m{2.5cm} % Coluna 5
		| >{\centering\arraybackslash}m{2.5cm} | % Coluna 6
	}
	{Valores de Aptidão dos Indivíduos da População Inicial}
	{%
			Índice & Indivíduo & Valor de Aptidão & Valor de Aptidão Acumulado & Participação em Relação à População & Participação Acumulada \\ \hline % Header
			1 & 11000 & 576 & 576 & 49,2\% & 49,2\% \\ \hline % Linha 1
			2 & 10011 & 361 & 937 & 30,9\% & 80,1\% \\ \hline % Linha 2
			3 & 01101 & 169 & 1106 & 14,4\% & 94,5\% \\ \hline % Linha 3
			4 & 01000 & 64 & 1170 & 5,5\% & 100\% \\ \hline % Linha 4
			Total & & 1170 & & 100\% & \\ \hline % Linha 5
	}
	{ValoresAptidaoIndividuosPopulacaoIndividual}
	{adaptado de \citet[p.11]{goldberg_genetic_1989}}

Com base no peso de cada indivíduo frente à população, o operador da reprodução pode ser aplicado através de uma roleta\footnote{O método da roleta é um dos vários possíveis na aplicação do operador de reprodução.} contendo 4 partes, cada parte relativa a um indivíduo, com proporções equivalentes às suas participações no total da população (\autoref{fig:RoletaReproducao}).

\figura
	{Roleta Enviesada para Aplicação do Operador de Reprodução}
	{.6}
	{imagens/RoletaReproducao.png}
	{RoletaReproducao}
	{adaptado de \citet[p.11]{goldberg_genetic_1989}}

A reprodução ocorre a cada girada na roleta. No caso do nosso exemplo, 4 vezes. Sendo assim, os indivíduos selecionados pelo operador de reprodução são copiados de forma ordenada, sem nenhuma alteração, para um reservatório temporário de acasalamento\footnote{Do inglês, \textit{mating pool}.} conforme vão sendo selecionados. Ou seja, os que tiverem maior valor de aptidão, logo maior chance de sobrevivência, possuem maiores chances de criarem um número maior de descendentes, passando parte de suas características para a próxima geração. Para o nosso exemplo, será utilizada uma taxa de reprodução de 100\%, com todos os indivíduos sendo replicados. Esta taxa pode variar conforme o problema de otimização a ser resolvido.

\subsection{Cruzamento}

No reservatório de acasalamento, os indivíduos irão combinar suas características entre si, processo que é também realizado de forma aleatória. Supondo que, ao girar a roleta, os indivíduos com maior valor de aptidão foram selecionados primeiro que os indivíduos com menor aptidão, formando, assim, os seguintes pares:

\tabelamultilinhas
	{p{1cm} c}
	{Pares Formados no Reservatório de Acasalamento}
	{%
		\multirow{2}{1cm} % Duas linhas;
			{Par 1} % Multilinhas 
				& Indivíduo 1: $11000$ \\ % Linha 1
				& Indivíduo 2: $10011$ \\ \hline \\ % Linha 2
		\multirow{2}{1cm} % Duas linhas;
			{Par 2} % Multilinhas
				& Indivíduo 3: $01101$ \\ % Linha 1
				& Indivíduo 4: $01000$ \\ \hline % Linha 2
	}
	{ParesFormadosReservatorioAcasalamento}
	{Elaborado pelo autor.}
        
Para aplicação do operador de cruzamento, assim como no operador de reprodução, também pode ser utilizada a roleta para o sorteio dos pontos onde os indivíduos serão divididos para cruzamento. Porém, agora, contendo quatro partes iguais, uma para cada ponto entre o locus de cada gene, assim como cada giro equivalerá a um cruzamento por par. Deste modo, ao girar a roleta, vamos supor que tenham sido sorteados os números 2 e 3. Todos os caracteres à direita dos pontos sorteados são trocados entre os indivíduos dos pares formados, o que será representado pelo símbolo separador \enquote{\textbar} (\autoref{tab:ParesFormadosReservatorioAcasalamentoSeparadoresCruzamento}) e, após o sorteio dos pontos de cruzamento, é realizada a recombinação dos pares (\autoref{tab:ParesFormadosReservatorioAcasalamentoAposCruzamento}).

\tabelamultilinhas
	{p{1cm} c}
	{Pares Formados no Reservatório de Acasalamento com Separadores de Cruzamento}
	{%
		\multirow{2}{1cm} % Duas linhas;
			{Par 1} % Multilinhas 
			& Indivíduo 1: $1.1|0.0.0$ \\ % Linha 1
			& Indivíduo 2: $1.0|0.1.1$ \\ \hline \\ % Linha 2
		\multirow{2}{1cm} % Duas linhas;
			{Par 2} % Multilinhas
			& Indivíduo 3: $0.1.1.0|1$ \\ % Linha 1
			& Indivíduo 4: $0.1.0.0|0$ \\ \hline % Linha 2
	}
	{ParesFormadosReservatorioAcasalamentoSeparadoresCruzamento}
	{Elaborado pelo autor.}

\tabelamultilinhas
	{p{1cm} c}
	{Pares Formados no Reservatório de Acasalamento Após o Cruzamento}
	{%
		\multirow{2}{1cm} % Duas linhas;
		{Par 1} % Multilinhas 
		& Indivíduo 1: $1.1|0.1.1$ \\ % Linha 1
		& Indivíduo 2: $1.0|0.0.0$ \\ \hline \\ % Linha 2
		\multirow{2}{1cm} % Duas linhas;
		{Par 2} % Multilinhas
		& Indivíduo 3: $0.1.1.0|0$ \\ % Linha 1
		& Indivíduo 4: $0.1.0.0|1$ \\ \hline % Linha 2
	}
	{ParesFormadosReservatorioAcasalamentoAposCruzamento}
	{Elaborado pelo autor.}

As novas subcadeias de caracteres, ou subvetores, são chamadas de blocos de construção. A Hipótese dos Blocos de Construção de um AG\footnote{Do inglês, \textit{Building block hypothesis}.} será aprofundada mais à frente. Por agora, fica-se com a analogia apresentada por \citeauthor{goldberg_genetic_1989}:

\citacao
	{%
		(...) considere uma população de $n$ cadeias de caracteres (...), onde cada uma é uma \textit{ideia} completa ou uma prescrição para realizar uma tarefa particular. As subcadeias de caracteres de cada cadeia de caracteres (ideia) contém várias \textit{noções} do que é importante ou relevante para a tarefa. (...) Então, a ação de cruzamento com reproduções prévias é especulada sobre novas ideias construídas através dos blocos de construção de alto desempenho (noções) de tentativas passadas. (...) A troca de noções para formar novas ideias é intuitiva se nós pensarmos em termos do processo de \textit{inovação}.
	}{%
	(...), consider a population of $n$ strings (...) over some appopriate alphabet, coded so that each is a complete \textit{idea} or prescription for performing a particular task (...). Substrings within each string (idea) contain various \textit{notions} of what is important or relevant to the task. Viewed in this way, the population contains not just a sample of $n$ ideas; rather, it contains a multitude of notions and rankings of those notions for task performance. (...) Thus, the action of crossover with previus reproduction speculates on new ideas constructed form high-performance building blockis (notions) of past trials. (...) Exchanging of notions to form new ideas is appealing intuitively, if we think in termos of the process of \textit{innovation}
	}
	{\citep[pg.13]{goldberg_genetic_1989}}
	{(tradução nossa).}

\subsection{Mutação}
\label{subsec:mutacao}

Após a aplicação dos operadores apresentados, ainda é necessário aplicar um último operador, o de mutação\footnote{\label{rodape:aplicacao_operador_mutacao}. O operador de mutação não precisa, necessariamente, ser aplicado na última parte do processo. Isso irá variar conforme as estratégias de construção do AG.}. Os operadores de reprodução e cruzamento mantêm as características, informação genética, dos indivíduos mais aptos, porém essa aptidão é relacionada exclusivamente à geração corrente. Dessa forma, o AG pode convergir para uma solução mais rápido que o desejado e perder genes importantes ao longo do processo (genes que estão localizados em locus específicos em cada indivíduo). Portanto, o operador de mutação é uma garantia secundária de que haja diversidade nas populações que serão geradas ao longo das gerações, permitindo que o AG procure por soluções mais robustas. Ou seja, através da paisagem de aptidão, o algoritmo irá buscar de forma mais ampla pelo maior pico possível (ótimo global), diminuindo as chances de apresentar como solução picos menores (ótimos locais).

Em relação ao processo em si, a mutação é tão simples quanto a reprodução e o cruzamento. Utilizando novamente a roleta, será escolhido aleatoriamente, levando em consideração uma taxa, qual indivíduo sofrerá ou não uma mutação. Iremos nos aprofundar nesse ponto futuramente, mas, a título de exemplo, vamos utilizar uma taxa de mutação de 1\%. Será realizado o giro da roleta para cada um dos indivíduos com o objetivo de selecionar qual ou quais sofrerão uma mutação. Supondo-se que o indivíduo 2 seja sorteado, temos:

 \figura
	{Seleção Aleatória de um Indivíduo para Mutação}
	{.8}
	{imagens/RoletaMutacao.png}
	{RoletaMutacao}
	{Elaborado pelo autor.}

Após a seleção dos indivíduos que sofreram a mutação, será definido, também de forma aleatória, em qual gene essa mutação ocorrerá. No caso do indivíduo 2, iremos supor que o gene no locus 2 foi o selecionado:

\figura
	{Mutação do Indivíduo Selecionado para Mutação}
	{.8}
	{imagens/RoletaMutacaoGene.png}
	{RoletaMutacaoGene}
	{Elaborado pelo autor.}

Como podemos observar no \autoref{tab:ValoresAptidaoIndividuosPopulacaoSegundaGeracao}, através dos processos de reprodução, cruzamento e mutação é gerada uma nova população, com indivíduos contendo novas características, onde a segunda geração melhorou em relação à primeira, vide o valor total de aptidão de 1530 em comparação aos 1170 da primeira geração, o que a torna uma solução incrementalmente melhor, sendo o indivíduo 1 o que possui as características que mais auxiliarão na resolução do problema da Caixa Preta.

\tabela
	{Valores de Aptidão dos Indivíduos da População Inicial e População na Geração Seguinte}
	{%
		\begin{tabular}{c|c|c|c|c|c|c|c} % Construção Tabela 1
			\cline{2-7}
			\multirow{7}{0.5cm}{\rotatebox{90}{População 1}}
			& \multicolumn{1}{ >{\centering\arraybackslash}m{1.2cm}| }{Índice} 
			& \multicolumn{1}{ >{\centering\arraybackslash}m{1.5cm}| }{Indivíduo} 
			& \multicolumn{1}{ >{\centering\arraybackslash}m{1.5cm}| }{Valor de Aptidão} 
			& \multicolumn{1}{ >{\centering\arraybackslash}m{2.5cm}| }{Valor de Aptidão Acumulado} 
			& \multicolumn{1}{ >{\centering\arraybackslash}m{2.5cm}| }{Participação em Relação à População}
			& \multicolumn{1}{ >{\centering\arraybackslash}m{2.5cm}| }{Participação Acumulada}
			& \\ 
			\cline{2-7} & 1 & 11000 & 576 & 576 & 49,2\% & 49,2\%  \\ 
			\cline{2-7} & 2 & 10011 & 361 & 937 & 30,9\% & 80,1\%  \\ 
			\cline{2-7} & 3 & 01101 & 169 & 1106 & 14,4\% & 94,5\%  \\ 
			\cline{2-7} & 4 & 01000 & 64 & 1170 &  5,5\% & 100\%  \\ 
			\cline{2-7} & Total & & 1170 & & 100\% & \\
			\cline{2-7} 
		\end{tabular}
		\begin{tabular}{cccccccc} % Construção Tabela em branco entre as duas tabelas com conteúdo
		\end{tabular}
		\begin{tabular}{c|c|c|c|c|c|c|c} % Construção Tabela 2
			\cline{2-7}
			\multirow{7}{0.5cm}{\rotatebox{90}{População 2}}
			& \multicolumn{1}{ >{\centering\arraybackslash}m{1.2cm}| }{Índice} 
			& \multicolumn{1}{ >{\centering\arraybackslash}m{1.5cm}| }{Indivíduo} 
			& \multicolumn{1}{ >{\centering\arraybackslash}m{1.5cm}| }{Valor de Aptidão} 
			& \multicolumn{1}{ >{\centering\arraybackslash}m{2.5cm}| }{Valor de Aptidão Acumulado} 
			& \multicolumn{1}{ >{\centering\arraybackslash}m{2.5cm}| }{Participação em Relação à População}
			& \multicolumn{1}{ >{\centering\arraybackslash}m{2.5cm}| }{Participação Acumulada}
			& \\ 
			\cline{2-7} & 1 & 11011 & 729 & 729 & 47,6\% & 47,6\%  \\ 
			\cline{2-7} & 2 & 11000 & 576 & 1305 & 37,6\% & 85,3\%  \\ 
			\cline{2-7} & 3 & 01100 & 144 & 1449 & 9,4\% & 94,7\%  \\ 
			\cline{2-7} & 4 & 01001 & 81 & 1530 &  5,3\% & 100\%  \\ 
			\cline{2-7} & Total & & 1530 & & 100\% & \\
			\cline{2-7} 
		\end{tabular}
	}
	{ValoresAptidaoIndividuosPopulacaoSegundaGeracao}
	{Elaborado pelo autor.}

Portanto, após apresentados todos os elementos da composição de um AG, assim como seus processos e operadores, fica mais claro como o algoritmo trabalhada para apresentar uma possível solução de um determinado problema. Claro, no exemplo da Caixa Preta, o indivíduo que apresentará as características que o definem como o mais apto, será o que contiver os genes $11111$, porém, não sabemos qual será a geração em que esse indivíduo, ou solução, será criado, assim como não foram consideradas restrições para aplicação do AG, assunto que será tratado no capítulo seguinte.
	% Algoritmos Genéticos
\chapter{Aplicação de Algoritmos Genéticos na Economia}

Neste capítulo, busca-se apresentar algumas aplicações de AGs em problemas na economia e discussões voltadas a contextos econômicos, especialmente, nas áreas de otimização, inovação e aprendizagem adaptativa.

\section{Introdução}

A economia, sendo uma ciência social, não possui um laboratório totalmente controlado para se testar hipóteses e modelos, havendo a necessidade de se utilizar premissas que tornem possível a aplicação matemática de conceitos e modelos teóricos em busca de encontrar respostas para as perguntas a serem respondidas sobre o comportamento dos agentes econômicos e da sociedade como um todo. Contudo, muitas vezes, os modelos se apresentam limitados ou insuficientes quando testados em problemas reais complexos.

A gama de áreas que compõe a economia no que diz respeito ao escopo de análises dos agentes, políticas públicas, conjecturas, entre outras, é vasta. Porém, pode-se dividi-la em dois ramos principais: macroeconomia e microeconomia. Conforme apresenta \citet[pg.3]{pindyck_microeconomia_2013}, a microeconomia estuda os agentes a nível individual, sendo trabalhadores, consumidores, investidores, empresas, etc. Neste nível, analisa-se como os agentes que participam dos processos econômicos tomam suas decisões, fazem suas escolhas ou interagem entre si, formando agentes econômicos maiores que apenas uma unidade isolada. Assim sendo, a microeconomia não busca observar indicadores agregados como a taxa de crescimento da economia, a inflação, o nível de desemprego, entre outros. Cabe à macroeconomia a análise destes agregados, e será nela que iremos nos concentrar para entender as formas de aplicação dos AGs em contextos econômicos.

O escopo de estudos da macroeconomia é apresentado por \citeauthor*{snowdon_modern_2005} da seguinte maneira:

\citacao
	{%
		A macroeconomia está preocupada com a estrutura, desempenho e comportamento da economia como um todo. A principal preocupação dos macroeconomistas é analisar e tentar entender os determinantes subjacentes das principais tendências agregadas da economia em relação à produção total de bens e servições (PIB), desemprego, inflação e transações nacionais. Em particular, a análise macroeconômica procura explicar a causa e o impacto das flutuações de curto prazo no PIB (o ciclo de negócios) e os principais determinantes da trajetória de longo prazo do PIB (crescimento econômico)
	}{%
		Macroeconomics is concerned with the structure, performance and behaviour of the economy as a whole. The prime concern of macroeconomists is to analyse and attempt to understand the underlying determinants of the main aggregate trends in the economy with respect to the total output of goods and services (GDP), unemployment, inflation and international transactions. In particular, macroeconomic analysis seeks to explain the cause and impact of short-run fluctuations in GDP (the business cycle), and the major determinants of the long-run path of GDP (economic growth).
	}
	{\citep[pg.1]{snowdon_modern_2005}}
	{(tradução nossa).}

Contudo, claro, os indicadores-objetivo da macroeconomia tem grande importância e impacto na vida das pessoas a nível individual. Uma economia que apresenta uma taxa de crescimento contínuo, e uma baixa taxa de desemprego e inflação, pode ser considerada bem sucedida em relação às suas políticas e estratégias econômicas que, por sua vez, impactarão positivamente o nível de felicidade e bem-estar da população como um todo. Em contrapartida, crises econômicas aumentam a insegurança e incerteza, assim como reduzem o nível geral de bem-estar e felicidade dos indivíduos. Dessa forma, as análises econômicas seguiram separadamente por um longo período, com uma alocação muito clara do que seria uma análise voltada à microeconomia ou macroeconomia, sendo a união dos conhecimentos e teorias destes dois campos de estudo relativamente recente (ver \autoref{sec:AsPrincipaisTeoriasEconomicasDoSeculoXXeXXI}).

Em termos de análises macroeconômicas, um dos principais objetivos, senão o principal, é a elaboração de políticas que visam manter uma estabilidade econômica frente aos choques endógenos e exógenos que podem afetar negativamente um conjunto de indicadores. Busca-se entender como determinados agregados, que compõem uma nação, se comportaram durante um período e como prevê-los no longo prazo como objetivo de antecipar diferentes tipos de cenários. Não há, contudo, uma forma correta, muito menos um consenso, de como se realizar tais análises, criando-se, ao longo do tempo, diferentes metodologias de como construir os argumentos em torno das análises conjecturais do passado, presente ou futuro. \citeauthor*{snowdon_modern_2005} ilustram muito bem a constituição de uma teoria econômica, onde, \enquote{organizada através de uma estrutura lógica (ou teórica), forma a base sobre a qual uma política econômica é projetada e implementada}, sendo que \enquote{[...] o problema intelectual para os economistas é como capturar, na forma de modelos específicos, o complicado comportamento interativo de milhões de indivíduos engajados na atividade econômica}\footnote{\textit{Macroeconomic theory, consisting of a set of views about the way the economy operates, organized within a logical framework (or theory), forms the basis upon which economic policy is designed and implemented. Theories, by definition, are simplifications of reality. This must be so given the complexity of the real world. The intellectual problem for economists is how to capture, in the form of specific models, the complicated interactive behaviour of millions of individuals engaged in economic activity.}} \citep[pg.4]{snowdon_modern_2005} (tradução nossa).

Assim, parte-se, inicialmente, de modelos simples, generalizados e com uma série de hipóteses em busca de cobrir toda a complexidade da realidade. Por sua vez, embora simples, estes modelos apresentam as primeiras respostas para o entendimento de como determinados fenômenos ocorrem e qual seu impacto na sociedade como um todo, podendo se considerar uma teoria bem sucedida, a que apresenta um conjunto de modelos que preveem determinados cenários e quais políticas econômicas poderão ser implementadas visando alcançar objetivos definidos de uma nação que, de maneira prática, apresentam-se como a busca por uma taxa de crescimento econômico satisfatória sem a desestabilização da economia durante o processo.

\section{As Principais Teorias Macroeconômicas do Século XX e XXI}
\label{sec:AsPrincipaisTeoriasEconomicasDoSeculoXXeXXI}

Em torno da pluralidade dos pensamentos e teorias econômicas ao longo do século XX e XXI, \citeauthor*{blanchard_macroeconomics_2020} \citeapudyear{blanchard_macroeconomics_2020}{snowdon_modern_2005}aponta que \enquote{o que os economistas acreditam hoje, é o resultado de um \textbf{processo evolutivo} (destacado por nós) no qual eles eliminaram aquelas ideias que falharam e mantiveram as que parecem explicar bem a realidade.}\footnote{\textit{What macroeconomists believe today is the result of an evolutionary process in which they have eliminated those ideas that failed and kept those that appear to explain reality well.}}

Há um certo consenso de que o entendimento da macroeconomia, como uma área "formal" de estudos, acontece na primeira metade do século XX. Contudo, por todo o período seguinte, houve discordâncias e controvérsias acerca de qual seria a melhor e mais correta metodologia para análises macroeconômicas. \Citet[pg.1-14]{mayer_1994}, em seu artigo intitulado \enquote{\textit{Why is there so much disagreement among economists?}}, lista sete causas principais para a falta de alinhamento entre os macroeconomistas ao longo da história: i) um conhecimento limitado do funcionamento da economia, ii) o conjunto de assuntos investigados em constante crescimento, iii) a necessidade de considerar demais fatores que causam grande impacto na economia que subdivide-se em mais quatro causas como iv) os fatores políticos; v) os julgamentos de valor; vi) as empatias sociais; e, por último, vii) as metodologias.

Ademais, como argumentam \Citet[pg.6-7]{snowdon_modern_2005}, a própria rigidez dos métodos de pesquisa científica convergem para um contexto de discussões e discordâncias, evitando que haja, por exemplo, uma estagnação no campo dos estudos macroeconômicos, assim como os embates se concentram mais nas questões teóricas, evidências apresentadas, ou empíricas, e das ferramentas utilizadas para resolução dos problemas, do que sobre quais são os reais objetivos das políticas econômicas (alto nível de emprego, estabilização dos preços e rápido crescimento econômico). Dessa forma, a escolha das ferramentas ou dos métodos que serão utilizados para resolver estes problemas dependerão dos diagnósticos das causas raízes dos problemas em uma economia. Neste ponto, divide-se os estudos em duas abordagens principais: clássica e keynesiana.

\subsection{Abordagem Clássica}

Dentre as diversas ideias apresentadas por \Citet{smith1996riqueza} em sua obra primordial da, futuramente conhecida como abordagem clássica, \enquote{A Riqueza das Nações: investigação sobre sua natureza e suas causas}, o \text{laissez-faire}\footnote{Expressão francesa que significa, de forma literal, \enquote{deixe estar ou deixe fazer}.} é a principal. Smith defendia que, em um cenário ideal, o governo deveria intervir o mínimo possível nos interesses e interações dos agentes privados, sobretudo, o comércio. Cabia ao governo, na figura do soberano à época, apenas três, importantes, deveres: i) \enquote{[...] o dever de proteger a sociedade contra a violência e a invasão de outros países independentes}; ii) \enquote{[...] o dever de proteger, na medida do possível, cada membro da sociedade contra a injustiça e a opressão de qualquer outro membro da mesma, ou seja, o dever de implementar uma administração judicial exata}; iii) \enquote{[...] o dever de criar e manter certas obras e instituições públicas que jamais algum indivíduo ou um pequeno contingente de indivíduos poderão ter o interesse de criar e manter [...]} \cite[pg.170]{smith2020riquezav2}.

Como apontado por \Citet[pg.105]{vasconcellos2015}, o período dos estudos econômicos referente à Escola Clássica (pré Keynes) é formado pela contribuição de diversos autores que contribuíram de forma isolada ao longo do século XIX e início do XX, tendo, em grande parte, as ideias de Smith como precursoras. Assim sendo, quando se fala da abordagem clássica, ou Escola Clássica, entende-se como uma compilação das principais contribuições pré keynesianismo.

Para a Escola Clássica, o mercado tende a sempre alcançar o equilíbrio econômico através do pleno emprego, ou seja, onde a oferta e a demanda por mão-de-obra são iguais. Assim, como existe uma predeterminação do nível de atividade e de emprego pelas forças de mercado, as chamadas variáveis reais, a variação dos preços é impactada apenas pela quantidade de moeda, isto é, não afeta os agregados econômicos nem os preços relativos. Com isso, valendo-se da máxima \enquote{a oferta cria sua própria demanda} (Lei de Say), as políticas monetárias não têm efeito sobre a oferta e demanda agregada (hipótese conhecida como a neutralidade da moeda\footnote{Na abordagem clássica há uma separação entre as variáveis reais (nível de emprego, produto, salário real, etc.) e as variáveis nominais ou monetárias (preços e salário nominal). A neutralidade da moeda parte da hipótese de que o estoque de moeda não afeta as variáveis reais, algo que ocorrerá apenas se existir alguma imperfeição ou distorção nos mercados.}), pois a demanda agregada não é um fator de impacto quando se olha para a oferta agregada que, por sua vez, é determinado pela produção total que as empresas e as famílias estão dispostas a ofertar por um determinado preço.

Dessa forma, partindo do pressuposto de que o mercado de trabalho é do tipo concorrência perfeita, ou seja, com um grande número de ofertantes (sem fixação dos preços por parte de sindicatos, por exemplo) e um grande número de demandantes (sem fixação dos salários que serão pagos pela empresa), o equilíbrio no mercado de trabalho é alcançado quando a demanda por mão-de-obra se iguala à oferta por mão-de-obra. Em outras palavras, num dado nível de salário real, a empresa encontrará uma oferta de trabalho suficiente para suprir a demanda e todos os trabalhadores que quiserem trabalhar, encontrarão um emprego. No caso de um excesso na oferta de trabalho, o resultado será a queda do salário real ($\left(\frac{W}{P}\right)_{1}$ para $\left(\frac{W}{P}\right)_{E}$) e um excesso na demanda de trabalho resultará no aumento do salário real ($\left(\frac{W}{P}\right)_{2}$ para $\left(\frac{W}{P}\right)_{E}$). Assim, o mercado sempre forçará um equilíbrio entre a oferta e a demanda por mão-de-obra (vide \autoref{fig:AbordagemClassicaEquilibrioMercadoTrabalho}).

\figuracomnota
	{Equilíbrio no Mercado de Trabalho}
	{.8}
	{imagens/AbordagemClassicaEquilibrioMercadoTrabalho.PNG}
	{AbordagemClassicaEquilibrioMercadoTrabalho}
	{adaptado de \citet[p.113]{vasconcellos2015}}
	{$N$: oferta de trabalho; $\frac{W}{P}$: salário real;}

Então, definido o nível de emprego de equilíbrio do mercado de trabalho e um determinado produto de pleno emprego, as variáveis que irão impactar na oferta agregada serão apenas as variáveis reais, ou seja, os fatores que, quando alterados, podem impactar no mercado de trabalho, como o aumento ou queda na produção marginal do trabalho (variação do acúmulo de capital ou na tecnologia), que pode afetar a demanda por mão-de-obra, ou um aumento populacional, por exemplo, que afeta a oferta por mão-de-obra. Em relação à demanda agregada, a oferta de moeda tenderá a ser sempre igual à demanda de moeda que, por sua vez, é proporcional à quantidade do produto total. Assim, para uma determinada quantidade de moeda ofertada, quanto maior o nível de preços, menor será o estoque de moeda disponível para satisfazer as transações em uma economia e, por consequência, também será menor a quantidade de bens e serviços a serem demandados. No caso de um aumento no estoque de moeda, a demanda agregada aumentará, e, no caso de uma queda, a demanda agregada irá reduzir.

Com isso, temos que o produto total de uma economia é afetado apenas pelas variáveis que influenciam a oferta agregada e que os fatores que influenciam a demanda agregada não impactam no produto total, já que afetam apenas os preços. O equilíbrio de pleno emprego, então, é alcançado através dos mecanismos de preços que operam a auto-regulação do mercado, evitando, assim, uma intervenção governamental em busca do equilíbrio.

\subsection{Abordagem Keynesiana}

A abordagem keynesiana é, sobretudo, uma negação da abordagem clássica. Como cita \citeauthor*{mankiw1989}, a ideia da auto-regulação dos agentes privados e intervenção mínima por parte do Estado frente às novas ideias apresentadas por Keynes de uma maior intervenção estatal em busca de regular a demanda agregada, visando manter um certo nível de produção e emprego, dominou as discussões econômicas ao longo do século XIX e XX:

\citacao
{%
	A Escola Clássica enfatiza a otimização dos atores econômicos privados, o ajuste dos preços para o equilíbrio da oferta e demanda e a eficiência dos mercados livres. A Escola Keynesiana acredita que o entendimento das flutuações econômicas requer, não apenas estudar os meandros do equilíbrio geral, mas também apreciar a possibilidade de falha de mercado em grande escala.
}{%
	The classical school emphasizes the optimization of private economic actors, the adjustment of relative prices to equate supply and demand, and the efficiency of unfettered markets. The Keynesian school believes that understanding economic fluctuations requires not just studying the intricacies of general equilibrium, but also appreciating the possibility of market failure on a grand scale.
}
{\citep[pg.79]{mankiw1989}}
{(tradução nossa).}

Segundo \Citet[pg.14]{snowdon_modern_2005}, Keynes via a teoria clássica como o grande problema da ineficiência de análises econômicas, à época, na resolução das dores sofridas pelas sociedades ao redor do mundo no período entre guerras, sendo desastrosa quando aplicada em problemas do mundo real. Para ele, o capitalismo não era a causa raiz, embora se apresentava como um sistema instável. Com isso, Keynes propôs novas ideias visando fortalecer este sistema por meio da diminuição ou eliminação completa das falhas mais pungentes do capitalismo através da busca pelo pleno emprego e estabelecendo as questões de eficiência, estabilidade e crescimento econômico como os temas centrais das políticas econômicas.

\section{Economia Evolucionária}

\section{Algoritmos Genéticos Aplicados a Problemas de Otimização}

\section{Algoritmos Genéticos Aplicados a Problemas de Inovação}

\section{Aprendizagem Adaptativa}

	% Aplicação de Algoritmos Genéticos na Economia
\chapter{Exemplos de Aplicações de Algoritmos Genéticos}

Neste capítulo, busca-se demonstrar a matemática, assim como sua representação programática, por trás de um AG para a resolução de um problema de otimização, levando em consideração os elementos apresentados no \autoref{chap:altoritmos_geneticos}. Para isso, utilizar-se-á uma adaptação do exemplo da Caixa Preta do livro \textit{Genetic Algorithms in Search, Optimization and Machine Learning}, de David E. Goldberg \citeyearpar{goldberg_genetic_1989}, bem como os processos realizados pelo AG serão construídos em Python, possibilitando maior entendimento prático através de exemplos computacionais em uma das linguagens de programação mais populares da atualidade\footnote{Segundo dados divulgados pelo site TIOBE \citet{noauthor_index_nodate}} (código disponibilizado no \autoref{pythoncode:ConstrucaoAlgoritmoGenetico}). Ademais, foi disponibilizado no \autoref{appx:TerminologiaNaturalComputacional} uma tabela sumarizando os principais termos e conceitos que serão demonstrados no presente capítulo contendo os termos dos sistemas naturais e seus equivalentes nos sistemas artificiais.

\section{Problema de Valor Máximo em Uma Caixa Preta}
\label{sec:ProblemaValorMaximoCaixaPreta}

Dada uma Caixa Preta \footnote{Ibid, p. \pageref{rodape:problema_caixa_preta}.} com 5 interruptores, onde cada interruptor possui um sinal de entrada 0 ou 1 (desligado ou ligado) e um único sinal de saída como resultado (recompensa) de uma configuração específica, será aplicado um AG para encontrar qual é a configuração que resultará no maior valor de recompensa, considerando o problema de maximização representado pela função

\begin{center}
	$f(x) = x^2$,
\end{center}

\noindent onde $x$ pode assumir qualquer inteiro no intervalo $\left[0, 31\right]$, função representada na \autoref{fig:GraficoFuncaoQuadraticaDeX}.

\figura
	{Gráfico da Função $\lowercase{f}(\lowercase{x}) = \lowercase{x}^2$}
	{1}
	{imagens/GraficoFuncaoQuadraticaDeX.png}
	{GraficoFuncaoQuadraticaDeX}
	{adaptado de \citet[pg.8]{goldberg_genetic_1989}}
	
\section{Construção da População Inicial}

O AG é, inicialmente, agnóstico em relação a maioria dos parâmetros que irá utilizar ao longo do processo de um problema de otimização. Como explicado por \citet[pg.8]{goldberg_genetic_1989}, não há a necessidade de trabalhar diretamente com o conjunto de parâmetros de entrada, como outros algoritmos de otimização podem demandar, sendo necessário apenas codificar os valores de entrada em valores binários.

Como visto anteriormente (ver \autoref{sec:algoritmos_e_linguagem_de_maquina}), o primeiro passo é codificar os parâmetros de entrada do algoritmo, o que será realizado através de um alfabeto binário representado por $V = \left\{0,1\right\}$. Assim, havendo 5 parâmetros de entrada (interruptores com sinal de desligado ou ligado), é possível construir uma representação através de um vetor, ou cadeia de caracteres, de 5 dígitos, que vai de $00000$ (inteiro $0$) a $11111$ (inteiro $31$). Esse vetor pode ser expressado como

\begin{center}
	$A_i = a_1a_2a_3a_4a_5 ... a_n$, \\
	FONTE: adaptado de \citet[pg.25]{goldberg_genetic_1989}
\end{center}

\noindent onde a notação $A$ representa um vetor (indivíduo) e $a$ representa uma característica específica (gene), subscrita por seu locus, contendo o alelo com valor $0$ ou $1$. As características não precisam, necessariamente, estarem ordenadas. Para ganho de desempenho, por exemplo, podem ser construídos vetores onde as características estejam fora de ordem em relação às suas posições originais, porém continuam com as subscrições de seus locus como, por exemplo, a variação do vetor anterior

\begin{center}
	$A'_i = a_3a_1a_5a_2a_4 ... a_n$, \\
	FONTE: adaptado de \citet[pg.25]{goldberg_genetic_1989}
\end{center}

\noindent sendo o apóstrofo (\textquotesingle) um símbolo de variação ou derivação de algum vetor específico.

Após a decodificação dos valores de entrada, é formada uma população inicial aleatória de tamanho $n$. Será considerada uma população inicial de tamanho $n = 5$ vetores, denotada por $\textbf{A}_j, j = 1, 2, ..., n$, onde o caractere maiúsculo e em negrito $\textbf{A}$ representa uma população num dado momento $t$ (iteração ou geração). A construção randômica dessa população será realizada através do lançamento de uma moeda honesta ($p_{cara} = p_{coroa} = 50\%$) para cada característica dos 5 vetores, ou seja, 25 lançamentos. Em outras palavras, para cada coroa sorteada, é adicionado um 0 na população e, para cada cara, um 1. Assim, têm-se os seguintes resultados baseados em uma simulação:

\tabelamulticolunas
	{c|c|c|c|c|c|c|c|c|}
	{População Inicial de Tamanho $\lowercase{n} = 5$}
	{%
		\cline{2-6}
		& \multicolumn{5}{|c|}{População Inicial} \\
		\cline{2-6}
		& \multicolumn{5}{|c|}{$\textbf{A}_1 = 0101100111101111110001011$} \\
		\cline{2-6}
		& Vetor $A_{11}$ & Vetor $A_{21}$ & Vetor $A_{31}$ & Vetor $A_{41}$ & Vetor $A_{51}$ \\
		\cline{2-6}
		& $01011$ & $00111$ & $10111$ & $11100$ & $01011$ \\
		\cline{2-6}
	}
	{GeracaoAleatoriaPopulacaoInicial}
	{Elaborado pelo autor.}

\section{Cálculo dos Valores de Aptidão}

Com a população inicial construída, é necessário decodificar os valores binários de cada vetor, o que pode ser feito através da fórmula

\begin{equation}
	{a_{ij}}^{l-1-i},
\end{equation}

\noindent onde $l$ é o tamanho do vetor e $a$ é o valor do alelo no locus $i$ da geração $j$.

Após a decodificação de cada valor binário, é feita a soma de todos os novos valores do vetor para, posteriormente, aplicar a função objetivo e ordenar os vetores conforme seus valores de aptidão (resultados apresentados na \autoref{tab:CalculoValoresAptidaoPopulacaoInicial}). Portanto, para o cálculo do valor total do vetor decodificado, aplica-se:

\begin{equation}
	x_n = \sum^{n}_{ij=1}{{a_{ij}}^{l-1-i}}
\end{equation}

\tabelamulticolunas
	{ccccccc}
	{Valores de Aptidão dos Vetores da População Inicial}
	{%
		& \multicolumn{1}{ >{\centering\arraybackslash}m{1.2cm} }{Nome Vetor} 
		& \multicolumn{1}{ >{\centering\arraybackslash}m{1.2cm} }{Vetor} 
		& \multicolumn{1}{ >{\centering\arraybackslash}m{2.8cm} }{Vetor com Alelos Decodificados} 
		& \multicolumn{1}{ >{\centering\arraybackslash}m{2.5cm} }{Valor de $x$}
		& \multicolumn{1}{ >{\centering\arraybackslash}m{2.5cm} }{Valor de Aptidão ($f(x) = {x_{ij}}^2$)}
		& \\ \cline{2-7}
		& $A_4$ & $11100$ & $\left[16,8,4,0,0\right]$ & 28 & 784 \\
		& $A_3$ & $10111$ & $\left[16,0,4,2,1\right]$ & 23 & 529 \\ 
		& $A_1$ & $01011$ & $\left[0,8,0,2,1\right]$ & 11 & 121 \\ 
		& $A_5$ & $01011$ & $\left[0,8,0,2,1\right]$ & 11 & 121 \\ 
		& $A_2$ & $00111$ & $\left[0,0,4,2,1\right]$ & 7 & 49 \\ 
		\cline{2-7}
		& & & Mínimo & 7 & 49 \\
		& & & Média & 16 & 320,8 \\
		& & & Soma & 80 & 1604 \\
		& & & Máximo & 28 & 784 \\
	}
	{CalculoValoresAptidaoPopulacaoInicial}
	{Elaborado pelo autor.}

\section{Operadores de Reprodução, Cruzamento e Mutação}

Calculados os valores de aptidão, o passo seguinte será criar uma nova geração de indivíduos através da aplicação dos operadores de reprodução, cruzamento e mutação.

\subsection{Reprodução}

Como visto na \autoref{subsec:reproducao_ou_selecao}, a reprodução é feita de forma aleatória levando em consideração uma probabilidade de seleção. Há diversas formas de \enquote{sortear} os vetores que serão selecionadas para o envio ao reservatório de acasalamento. Para este exemplo, a taxa de reprodução da população será de 100\%, ou seja, 5 vetores reproduzidos, e será aplicado o método da roleta para sorteio dos indivíduos que serão selecionados para o reservatório. A probabilidade de reprodução de cada indivíduo será igual ao peso do seu valor de aptidão em relação à população, assim como será calculada a probabilidade de seleção esperada como métrica de comparação, sendo a probabilidade calculada por

\begin{equation}
	p_{s}(x_{ij}) = \frac{f(x_{ij})}{\sum^{n}_{ij=1}{f(x_{ij})}}
\end{equation}

\noindent e a probabilidade esperada por

\begin{equation}
	E_{s}(x_{ij}) = \frac{f(x_{ij})}{{\overline{f}(x_{ij})}}
\end{equation}

Como é possível observar na \autoref{tab:CalculoOperadorReproducao}, a simulação através do método da roleta resultou em 3 reproduções do Vetor $A_4$, 1 reprodução dos Vetores $A_3$ e $A_5$ nenhuma reprodução de $A_1$ e $A_2$. O resultado foi próximo ao esperado, com um pequeno desvio no Vetor $A_3$, que estava mais próxima de 2 reproduções do que uma, e no Vetor $A_4$, que estava mais próxima de 2 reproduções do que as 3 resultantes. Contudo, pode-se observar que os resultados seguiram as ideias apresentadas até o momento, onde os indivíduos que possuem maior valor de aptidão em relação ao ambiente, têm uma tendência a serem selecionados para reprodução mais vezes, assim como os que possuem valores baixos de aptidão têm chances de serem selecionados poucas vezes ou não serem selecionados, o que significa a morte destes indivíduos.

\tabelamulticolunas
	{ccccccccc}
	{Resultados Após a Reprodução dos Vetores da População Inicial}
	{%
		& \multicolumn{1}{ >{\centering\arraybackslash}m{1cm} }{Nome Vetor} 
		& \multicolumn{1}{ >{\centering\arraybackslash}m{1.2cm} }{Vetor} 
		& \multicolumn{1}{ >{\centering\arraybackslash}m{1.2cm} }{Valor de $x$}
		& \multicolumn{1}{ >{\centering\arraybackslash}m{2.3cm} }{Valor de Aptidão ($f(x) = {x_i}^2$)}
		& \multicolumn{1}{ >{\centering\arraybackslash}m{2.3cm} }{Probabilidade de Reprodução}
		& \multicolumn{1}{ >{\centering\arraybackslash}m{2cm} }{Nº Esperado de Seleções}
		& \multicolumn{1}{ >{\centering\arraybackslash}m{2cm} }{Nº de Seleções (Roleta)}
		& \\ \cline{2-8}
		& $A_4$ & $11100$ & 28 & 784 & 48,9\% & 2,44 & 3 \\
		& $A_3$ & $10111$ & 23 & 529 & 33\% & 1,65 & 1 \\ 
		& $A_1$ & $01011$ &  11 & 121 & 7,5\% & 0,38 & 0 \\ 
		& $A_5$ & $01011$ & 11 &  121 & 7,5\% & 0,38 & 1 \\ 
		& $A_2$ & $00111$ & 7 & 49 & 3,1\% & 0,15 & 0 \\ 
		\cline{2-8}
		& ~ & Mínimo & 7 & 49 & 0,7\% & 0,15 & 0 \\
		& ~ & Média & 16 & 320,88 & 20\% & 1,00 & 1,00 \\
		& ~ & Soma & 80 & 1604 & 100\% & 5,00 & 5,00 \\
		& ~ & Máximo & 28 & 784 & 36,9\% & 2,44 & 3,00 \\
	}
	{CalculoOperadorReproducao}
	{Elaborado pelo autor.}

\subsection{Cruzamento}

O passo seguinte, é a aplicação do operador de cruzamento entre os indivíduos que foram copiados para o reservatório de acasalamento. Este processo será feito em 3 partes, sendo elas: a formação aleatória de pares; a escolha aleatória dos pontos de cruzamento; e, por último, o cruzamento entre os pares e a construção das novos vetores.

\subsubsection{Formação dos pares}

Como foram replicados 5 vetores para o reservatório de acasalamento, se for seguida uma premissa de pares exclusivos, um dos vetores não encontrará um par, logo, não será capaz de efetivamente criar um descendente. Há várias estratégias que podem ser seguidas para lidar com esse ponto. Em relação à população construída no exemplo, será considerada a premissa de que todos os indivíduos façam parte de, ao menos, um par. Ou seja, haverá pelo menos um indivíduo que fará par com dois outros indivíduos da população. 

Para a formação dos pares, será sorteado um vetor em seguida do outro, sendo que o último vetor selecionado formará par com o selecionado anteriormente. Estes, vão sendo retirados como opções no sorteio à medida que forem sendo escolhidos. No caso do último vetor, que ficaria sem par nesse processo, será realizado um novo sorteio para seleção de um dos 4 vetores já selecionados, resultando, assim, na formação de 3 pares para cruzamento. A probabilidade de seleção pode ser representada pela fórmula recursiva

\textcolor{red}{
	\begin{equation}
		p_{c}(x_{n}) = {
			\begin{cases}
				1 &\text{se $n-1 = 1$} \\
				\frac{1}{n-1} & \text{se $1 < n \leq n-1$}
			\end{cases}
		}
	\end{equation}
}

\subsubsection{Escolha do ponto de cruzamento}

Com a formação dos pares para cruzamento, será sorteado em qual ponto do vetor ocorrerá a troca de informações. Como ilustrado por \citet[p.12]{goldberg_genetic_1989}, este processo é bastante simples: escolhida uma posição $k$ de forma aleatória, os vetores pares trocarão todas as informações da posição $k + 1$ a $l$. Cada posição $k$, é localizada entre os valores binários e representada por um inteiro no intervalo $\left[1, l - 1\right]$, com probabilidade de escolha aleatória do ponto de cruzamento definida por

\begin{equation}
	p_{k}[x_{1}, x_{2}] = \frac{1}{{l - 1}}
\end{equation}

\subsubsection{Formação dos novos vetores}

Por último, definido o ponto de cruzamento, os vetores trocam informações entre si, formando vetores filhos. Em outras palavras, utilizando como exemplo os Vetores $A_4 = 11100$ e $A_3 = 10111$, supondo um valor sorteado de $k = 2$ no intervalo $\left[1, 4\right]$ , todos os valores binários do locus 3 ao 5 serão trocados entre os dois vetores que formam o par, resultando nos vetores $A'_4 = 11111$ e $A'_3 = 10100$, que farão parte da geração seguinte. Os resultados da aplicação do operador de cruzamento nos indivíduos enviados para o reservatório de cruzamento são apresentados na \autoref{tab:FormacaoParesSorteioPontoCruzamento}.

Como é possível observar, utilizando a premissa de que todos os indivíduos fizessem parte de ao menos um par, o Vetor $A_1$ foi sorteado como par do Vetor $A_0$ e do Vetor $A_4$, Par 2 e Par 3 respectivamente. Para o Par 1, foi selecionado o ponto de cruzamento $k = 3$ localizado entre o locus 3 e 4 e para os Pares 2 e 3, foi selecionado o ponto $k = 2$, entre os locus 2 e 3. Na \autoref{tab:VetoresAposAplicacaoOperadorCruzamento}, é possível observar os vetores após o cruzamento entre os pares.

\tabelamulticolunas
	{ccccccc}
	{Formação dos Pares e Sorteio do Ponto de Cruzamento}
	{
		& \multicolumn{1}{ >{\centering\arraybackslash}m{1.2cm} }{Id Par} 
		& \multicolumn{1}{ >{\centering\arraybackslash}m{2cm} }{Id Vetores} 
		& \multicolumn{1}{ >{\centering\arraybackslash}m{2.8cm} }{Vetores} 
		& \multicolumn{1}{ >{\centering\arraybackslash}m{2.5cm} }{Ponto de Cruzamento Sorteado}
		& \multicolumn{1}{ >{\centering\arraybackslash}m{3cm} }{Vetores com Ponto de Cruzamento}
		& \\ \cline{2-7}
		& Par 1 & $\left[A_2, A_3\right]$ & $\left[11100, 10111\right]$ & 3 & $\left[1.1.1|0.0, 1.0.1|1.1\right]$ \\
		& Par 2 & $\left[A_1, A_0\right]$ & $\left[11100, 11100\right]$ & 2 & $\left[1.1|1.0.0, 1.1|1.0.0\right]$ \\ 
		& Par 3 & $\left[A_4, A_1\right]$ & $\left[01011, 11100\right]$ & 2 & $\left[0.1|0.1.1, 1.1|1.0.0\right]$ \\
		\cline{2-7}
	}
	{FormacaoParesSorteioPontoCruzamento}
	{Elaborado pelo autor.}

\tabelamulticolunas
	{ccccccc}
	{Vetores Após Aplicação do Operador de Cruzamento}
	{%
		& \multicolumn{1}{ >{\centering\arraybackslash}m{1.2cm} }{Id Par} 
		& \multicolumn{1}{ >{\centering\arraybackslash}m{2cm} }{Id Vetores Pais} 
		& \multicolumn{1}{ >{\centering\arraybackslash}m{2.8cm} }{Vetores Pais} 
		& \multicolumn{1}{ >{\centering\arraybackslash}m{2.5cm} }{Id Vetor Filho (Após Cruzamento))}
		& \multicolumn{1}{ >{\centering\arraybackslash}m{3cm} }{Vetor Filho (Após Cruzamento)}
		& \\ \cline{2-7}
		& Par 1 & $\left[A_2, A_3\right]$ & $\left[11100, 10111\right]$ & $A'_1$ & $11111$\\
		& Par 1 & $\left[A_2, A_3\right]$ & $\left[11100, 10111\right]$ & $A'_2$ & $10100$ \\ 
		& Par 2 & $\left[A_1, A_0\right]$ & $\left[11100, 11100\right]$ & $A'_3$ & $11100$ \\
		& Par 2 & $\left[A_1, A_0\right]$ & $\left[11100, 11100\right]$ & $A'_4$ & $11100$ \\
		& Par 3 & $\left[A_4, A_1\right]$ & $\left[01011, 11100\right]$ & $A'_5$ & $01100$ \\ 
		& Par 3 & $\left[A_4, A_1\right]$ & $\left[01011, 11100\right]$ & $A'_6$ & $11011$ \\
		\cline{2-7}
	}
	{VetoresAposAplicacaoOperadorCruzamento}
	{Elaborado pelo autor.}

\subsection{Mutação}

O operador final a ser aplicado, é o de mutação\footnote{Ibid, p. \pageref{rodape:aplicacao_operador_mutacao}.}. Como apresentado na \autoref{subsec:mutacao}, a mutação é um recurso secundário para evitar que o algoritmo retorne picos locais, ao invés do pico global, como resultado da busca. A taxa de mutação é aplicada bit a bit, o que, em outras palavras, significa que cada elemento da característica de um indivíduo possui uma probabilidade bem pequena de mudança devido à pressão do ambiente. Para o presente exemplo, será considerada a probabilidade de mutação apresentada por \citet[pg.25]{goldberg_genetic_1989} de 0,1\%, ou:

\begin{equation}
	p_{m}(a_{ij}) = 0,001
\end{equation}

\noindent e o número esperado de bits que sofrerão mutação por

\begin{equation}
	E_{m}(a_{ij}) = n \cdot l \cdot p_{m}(a_{ij})
\end{equation}

Esperava-se que 0.025 ($5 \cdot 5 \cdot 0,001$) bits sofressem mutação, ou seja, nenhum. Após a aplicação do operador de mutação nos dados simulados, como esperado, nenhum dos 30 bits sofreu mutação.
	% Exemplos de Aplicações de Algoritmos Genéticos
\input{./capitulos/capitulo-6.tex}	% Conclusão

\addcontentsline{toc}{chapter}{Referências}
\bibliography{auxiliar/referencias}


\appendixpage
\begin{appendices}

\chapter{Sumário de Terminologia Natural e Terminologia Artificial}
\label{appx:TerminologiaNaturalComputacional}
\tabelalonga
	{%
		| >{\centering\arraybackslash}m{0.15\textwidth}
		| >{\centering\arraybackslash}m{0.15\textwidth}
		| >{\centering\arraybackslash}m{0.15\textwidth}
		| >{\centering\arraybackslash}m{0.50\textwidth} |
		}
	{%	
		Terminologia Natural &   Terminologia Artificial & Terminologia Artificial (Inglês) & Descrição Curta \\ \hline
		Locus & Locus &   \textit{Locus} & Posição de um gene (ou característica) dentro de uma cadeia de caracteres. \\ \hline
		Alelo & Alelo &   \textit{Alelle} & Valor do gene (ou característica), sendo 0 ou 1. \\ \hline
		Gene & Característica &   \textit{Gene} & Característica, caractere ou algoritmo de uma cadeia de caracteres. \\ \hline
		Bloco de construção & Bloco de Construção &   \textit{Building Block} & Subvetor de alto desempenho. Responsável por formar novos vetores. \\ \hline
		Cromossomo ou Indivíduo & Vetor ou Cadeia de caracteres &   \textit{String} & Conjunto de características agrupadas através de um conjunto de caracteres. Cada matriz pode ser uma possível solução para um problema ou fazer parte de uma solução global. \\ \hline
		Genótipo & Estrutura &   \textit{Structure} & Conjunto de vetores que mantém uma carga ou estrutura genética. \\ \hline
		Fenótipo & Estrutura decodificada &   \textit{Decoded structure} & Nova carga ou estrutura genética após as alterações das características principais de uma \\ \hline
		População & População &   \textit{Population} & Conjunto de indivíduos construídos, inicialmente, de forma aleatória, onde o AG irá realizar suas buscas. Cada geração é composta por uma população que pode ser uma possível solução para o problema de otimização. \\ \hline
		Geração & Iteração & \textit{Iteration} & Formação de uma nova população após a aplicação dos operadores de reprodução, cruzamento e mutação. \\ \hline 
		Reprodução ou Seleção & Operador de Reprodução & \textit{Reproduction} & Seleção dos indivíduos com maior aptidão ao ambiente. \\ \hline 
		Cruzamento & Operador de Cruzamento & \textit{Crossover} & Troca de genes, ou características, entre os indivíduos de uma população. \\ \hline 
		Mutação & Operador de Mutação & \textit{Mutation} & Processo aleatório de alteração de uma ou mais características de um indivíduo. \\ \hline 
		Valor de Aptidão & Valor de Aptidão & \textit{Fitness value} & Quantificação da qualidade de um indivíduo ou sua aptidão em relação ao ambiente. \\ \hline
		Paisagem ou Horizonte de aptidão & Paisagem de Aptidão & \textit{Fitness landscape} & Espaço pelo qual o AG faz sua busca por soluções, ou seja, diz respeito ao problema a ser resolvido. \\ \hline
		Função de aptidão & Função objetivo & \textit{objective function} & Função contendo os parâmetros para resolução do problema. \\ \hline
	}

\chapter{Exemplo de Aplicação de um AG simples em Python}
\label{pythoncode:ConstrucaoAlgoritmoGenetico}

\pythonexternal{auxiliar/monolito_algoritmo_genetico.py}

\end{appendices}
\end{document} 









